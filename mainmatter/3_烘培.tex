\chapter{烘培 \& 甜品}

文中如没有额外提到,默认烤箱均需要预热


\section{法式千层酥皮}
\begin{table}[h]
    \centering
    \begin{tabular}{|l||c|c|c|}\hline
     \textbf{材料}    &  \textbf{万用酥皮}& \textbf{甜品酥皮} &\textbf{传统酥皮}\\ \hline\hline
    \multicolumn{4}{|c|}{面皮}\\ \hline
    中筋面粉 & 240g & 200g& 600g\\ \hline
    盐 & 5g & 3g& 10g\\ \hline
    无盐黄油 & 25g & 20g& 130g\\ \hline
    冷水 & 125g & 110g& 250g\\ \hline
    糖 &  & 15g& 12g\\ \hline
    柠檬汁/醋 &  & & 15g\\ \hline
    \multicolumn{4}{|c|}{黄油团}\\ \hline
    中筋面粉 & 10g & 50g& 150g\\ \hline
    无盐黄油 & 180g & 150g& 675g\\ \hline
    \end{tabular}
    \caption{材料表}
\end{table}

\begin{enumerate}
    \item 制作细节待补充

\end{enumerate}


\section{戚风蛋糕}

\begin{table}[h]
    \centering
    \begin{tabular}{|l||c|c|c|}\hline
     \textbf{材料}    &  \textbf{6寸份量}& \textbf{8寸份量} &\textbf{10寸份量}\\ \hline\hline
    \multicolumn{4}{|c|}{蛋黄糊}\\ \hline
    蛋黄 & 3个 & 5个& 9个\\ \hline
    牛奶 & 42g & 70g& 126g\\ \hline
    玉米油 & 39g & 65g& 117g\\ \hline
    细砂糖 & 18g & 30g& 54g\\ \hline
    低筋面粉 & 54g & 90g& 162g\\ \hline
    玉米粉 & 6g & 10g& 18g\\ \hline
    \multicolumn{4}{|c|}{蛋白霜}\\ \hline
    蛋白 & 3个 & 5个& 9个\\ \hline
    柠檬汁 & \multicolumn{3}{|c|}{几滴}\\ \hline
    细砂糖 & 27g & 45g& 81g\\ \hline
    \end{tabular}
    \caption{材料表}
\end{table}

\begin{enumerate}
    \item 制作蛋黄糊。
    \begin{enumerate}
        \item 将牛奶,玉米油和糖放入碗中,搅拌均匀至糖完全融化,油和牛奶彻底混合。
        \item 筛入低筋面粉和玉米粉,搅拌至没有颗粒,不可用力过度搅出筋。
        \item 加入蛋黄,搅拌至完全均匀融合。不可以用力过度。
    \end{enumerate}
    \item 制作蛋白霜。将蛋白放入另一个料理盆中,加入几滴柠檬开始最高速打发。出大泡时加入3分之1糖,大泡都变小泡后加入3分之1糖,最后变细腻后加入剩余糖。打发至干性发泡。
    \item 将3分之1的蛋白霜放入蛋黄糊的碗里,用刮刀翻拌均匀。
    \item 再将搅拌好的蛋黄糊倒入蛋白霜中,翻拌均匀。
    \item 将搅拌好的面糊抬至离模具有一定高度的位置倒入,这样使其分布均匀,也能震出空气。
    \item 轻摔几下模具,震出气泡,放入预热好的烤箱里中下层开始烤。6寸模具需要145度55分钟,8寸模具需要145度60分钟,10寸模具需要155度60分钟。根据自身烤箱情况,时间上下误差在10分钟内。
    \item 烤好后马上拿出轻摔震几下,然后倒扣2小时以上即可脱模。

\end{enumerate}


\section{肉桂卷}

\begin{table}[h]
    \centering
    \begin{tabular}{|l||c|}\hline
     \textbf{材料}    &  \textbf{份量}\\ \hline\hline
    \multicolumn{2}{|c|}{面团}\\ \hline
    牛奶     &  490 g \\ \hline
    细砂糖     &  100.5g \\ \hline
    无盐黄油     &  113.5g \\ \hline
    耐高糖活性干酵母  & 10g  \\ \hline
    中筋面粉 &  500g  \\ \hline
    泡打粉     &  4.2g \\ \hline
    盐     & 8.4g  \\ \hline
    \multicolumn{2}{|c|}{填充}\\ \hline
    红砂糖     &  146.25g \\ \hline
    无盐黄油     &  170.25g \\ \hline
    肉桂粉     &  11.38g \\ \hline
    \multicolumn{2}{|c|}{糖霜}\\ \hline
    奶油奶酪     &  114g \\ \hline
    无盐黄油     &  28.38g \\ \hline
    牛奶     &  61.24g\\ \hline
    香草精     &  4.2g\\ \hline
    糖粉     &  121g \\ \hline
    \end{tabular}
    \caption{材料表}
\end{table}

\begin{enumerate}
    \item 先做面团:把热牛奶,细砂糖和已融化的黄油混合,然后加入酵母,等待10分钟以上开始活化产生气泡。
    \item 加入面粉进行混合。均匀揉成团后,盖上布,放置温暖地方发酵1小时,直到变成两倍大。
    \item 等待期间做填充料:将红砂糖,融化的无盐黄油和肉桂粉混合即可。填充可以加入朗姆酒渍葡萄干,核桃仁等等。
    \item 面团醒好后,加入剩余的面粉,盐和泡打粉。然后进行大力揉。不一定要出膜。保持面团软,一点点湿是允许的。最少揉10分钟。直到光滑不粘,手指按下去后会回弹。
    \item 将面团擀成长方形,厚度1厘米。把填充均匀铺上。然后卷起来。用绳子或牙线进行切割。大约8厘米宽,如果薄的话会感觉像饼干,并不合适。
    \item 将模具用黄油擦一遍防粘。把肉桂卷面团放在里面后,用布盖上放在温暖的地方,进行二次发酵,需要35到45分钟,直到增加一半大小。
    \item 预热烤箱180度。 然后将肉桂卷放入烤25-30分钟,直到棕黄色。
    \item 烤制时可以做糖霜,将除了糖粉外的所有材料混合在一起,搅拌均匀后,加入糖粉。搅拌。
    \item 烤出的肉桂卷要等10分钟以上,表面没有那么烫的时候再倒糖霜,以免融化太快。
\end{enumerate}

\section{Panna Cotta}

\begin{table}[h]
    \centering
    \begin{tabular}{|l||c|}\hline
     \textbf{材料}    &  \textbf{份量}\\ \hline\hline
    牛奶     &  250ml \\ \hline
    淡奶油     &  250ml \\ \hline
    砂糖     &  30g \\ \hline
    吉利丁片  & 7g  \\ \hline
    香草荚 &  半支  \\ \hline
    如果不用香草,想用咖啡,抹茶粉 &  8g  \\ \hline
    \end{tabular}
    \caption{材料表}
\end{table}

制作咖啡/抹茶奶冻时,步骤2略有差异,事先将粉类过滤,可在牛奶微沸时,倒出一小部分直接融化咖啡/抹茶粉,再过滤回锅中。抹茶粉比较难溶解,在没有均质机的情况下需多多耐心过筛以保证无结块

\begin{enumerate}
    \item 将7g吉利丁片用冷水浸泡至软化备用
    \item 250ml牛奶倒入锅中,切开香草荚,将香草籽刮入牛奶中,中小火煮至微沸,持续搅拌不要让牛奶出现奶皮
    \item 加入30g砂糖,不停搅拌,防止砂糖沾底。直至所有糖融化
    \item 倒入250ml淡奶油,搅拌均匀,待微沸关火
    \item 持续搅拌,待溶液稍事冷却后,沥干事先泡软的7g吉利丁片水分,加入溶液,搅拌至完全溶解
    \item 连容器隔冰水冷却,时不时搅拌防止表面产生奶皮形成杂质。如果嫌麻烦可以等放凉后,捞掉奶皮过滤即可,只是会损失部分香草籽。在这里之所以使用冰水,是为了让香草奶冻快速凝固。如果直接灌入容器,会使香草籽全部沉底而丧失风味。略微凝固后用刮刀搅拌均匀,使香草籽均匀分布后再灌入容器中
    \item 大致可以装90g的容器5-6个,震掉气泡,密封放入冰箱冷藏4小时以上。即可享用
\end{enumerate}





\section{提拉米苏}

\begin{table}[H]
    \centering
    \begin{tabular}{|l||c|}\hline
     \textbf{材料}    &  \textbf{份量}\\ \hline\hline
    马斯卡彭奶油奶酪 & 250g \\ \hline
    淡奶油 &  150g \\ \hline
    蛋黄 & 2 \\ \hline
    细砂糖 & 75g \\ \hline
    咖啡酒 & 50ml \\ \hline
    柠檬汁 & 5g \\ \hline
    手指饼干 & 根据模具判断 \\ \hline
    可可粉 & 适量 \\ \hline
    糖粉 & 可选,仅装饰 \\ \hline
    \end{tabular}
    \caption{材料表}
\end{table}

\begin{enumerate}
    \item 将蛋黄和细砂糖放入碗中,准备一锅水开最小火,隔水加热搅拌,直到细砂糖溶解,蛋黄发白,整体丝滑。注意此过程中水不能沸,以免蛋黄蒸熟
    \item 将马斯卡彭打发至顺滑
    \item 将淡奶油打发成至7成发
    \item 将打发后的奶油,马斯卡彭和蛋黄糊混合搅拌均匀成芝士陷(先蛋黄糊和马斯卡彭混合,再将其与奶油混合)
    \item 将手指饼干与咖啡酒中浸泡一会,正方面都蘸上咖啡酒
    \item 将芝士馅铺一层至模具底部,然后摆一层手指饼干,以此方法最终会有三层芝士馅,中间两层手指饼干即可。放置冰箱冷藏10小时。
\end{enumerate}


\section{翻转苹果塔}

\begin{table}[h]
    \centering
    \begin{tabular}{|l||c|}\hline
     \textbf{材料}    &  \textbf{份量}\\ \hline\hline
    \multicolumn{2}{|c|}{面皮}\\ \hline
    低筋面粉+高筋面粉     &  250 g \\ \hline
    无盐黄油     &  160g (实际用100g即可)\\ \hline
    牛奶  & 50ml  \\ \hline
    盐 &  5g  \\ \hline
    糖     &  15g \\ \hline
    蛋黄     & 1个  \\ \hline
    \multicolumn{2}{|c|}{陷}\\ \hline
    苹果     &  1000g \\ \hline
    糖     &  200g \\ \hline
    黄油     &  75g \\ \hline
    \end{tabular}
    \caption{材料表}
\end{table}

\begin{enumerate}
    \item 将面皮材料混合成面团后,揉至均匀,放置冰箱冷藏2小时
    \item 苹果切片或小滚刀块
    \item 锅里下黄油小火,半融化后倒糖。如果糖无法正确融化到棕色,加微量水直到变成棕色。接着倒苹果。煮20分钟以上 (直到苹果软化,不再出水)
    \item 面团擀成皮,将面皮直接盖在锅里,包裹所有陷,边缘部分需要多点面皮。中间用叉子戳小洞透气。
    \item 180度上下火,中层烤25分钟
\end{enumerate}

\section{蛋挞}

\begin{table}[h]
    \centering
    \begin{tabular}{|l||c|}\hline
     \textbf{材料}    &  \textbf{份量}\\ \hline\hline
    \multicolumn{2}{|c|}{蛋挞酥皮}\\ \hline
    低筋面粉     &  220 g \\ \hline
    无盐黄油     &  180g (裹入,若没有酥皮制作经验,可以降低黄油用量至150g)\\ \hline
    高筋面粉  & 30g  \\ \hline
    无盐黄油 &  40g  \\ \hline
    细砂糖     &  5g \\ \hline
    盐     & 1.5g  \\ \hline
    水     & 125g  \\ \hline
    \multicolumn{2}{|c|}{蛋挞水}\\ \hline
    淡奶油     &  180g \\ \hline
    牛奶     &  140g \\ \hline
    蛋黄     &  4个 \\ \hline
    细砂糖 & 80g  \\ \hline
    低筋面粉 & 15g \\ \hline
    炼乳 & 15g \\ \hline
    \end{tabular}
    \caption{材料表}
\end{table}

\begin{enumerate}
    \item 此酥皮配方与法式千层酥皮功能一样,简化了黄油团,将黄油直接通过冷藏裹入。也可直接用法式酥皮配方替代。
    \item 制作酥皮:
    \begin{enumerate}
        \item 面粉和糖、盐混合,将40克黄油放于室温使其软化,加入面粉中
        \item 倒入清水,揉成面团。水不要一次全部倒入,而需要根据面团的软硬度酌情添加
        \item 揉成光滑的面团。用保鲜膜包好,放进冰箱冷藏松弛20分钟
        \item 把180克裹入用的黄油切成小片,放入保鲜袋排好
        \item 用擀面杖把黄油压成厚薄均匀的一大片薄片。这时候黄油会有轻微软化,放入冰箱冷藏至重新变硬
        \item 把松弛好的面团取出来,案上施一层防粘薄粉,把面团放在案板上,擀成长方形,长大约为黄油薄片宽度的三倍,宽比黄油薄片的长度稍宽一点
        \item 把冷藏变硬的黄油薄片取出来,撕去保鲜袋,把黄油薄片放在长方形面片中央
        \item 把面片的一端向中央翻过来,盖在黄油薄片上。把面片的另一端也放过来,这样就把黄油薄片包裹在面片里了
        \item 把面片的一端压死,手沿着面片一端贴着面皮向另一端移过去,把面皮中的气泡从另一端赶出来,避免把气泡包在面片里。最终手移到另一端时,把另一端也压死
        \item 用擀面杖再次擀成长方形。擀的时候,由中心向四个角的方向擀,容易擀成规则的长方形
        \item 将面皮的一端向中心折过来, 将面皮的另一端也向中心翻折过来, 再把折好的面皮对折。这样就完成了第一轮的4折。 四折好的面片,包上保鲜膜,放入冰箱冷藏松弛20分钟左右
        \item 松弛好的面片拿出来,重复以上步骤,再进行4折两轮。一共进行3轮4折
        \item 把三轮4折完成的面片擀开成厚度约0.3CM的长方形。千层酥皮就做好了
    \end{enumerate}
    \item 制作蛋挞水:淡奶油与牛奶混合,加入细砂糖与炼乳,加热搅拌至砂糖溶解。冷却至不烫手后,加入蛋黄与低筋面粉,搅拌均匀即可
    \item 将千层酥皮擀成0.3cm的长方形,沿着一边卷起来,放置冰箱冷藏10分钟
    \item 取出冷藏好的酥皮,用刀切成1cm的剂子
    \item 将剂子的一面在面粉上沾一下。并将齐放入蛋挞模中,沾面粉的一面朝上。用手指将剂子捏成蛋挞模形状后静置20分钟
    \item 导入蛋挞水,7分满即可
    \item 220度上小火中层25分钟。(注意预热)
\end{enumerate}
此配方可做24个蛋挞




\section{蛋黄油酥饼干}

\begin{table}[H]
    \centering
    \begin{tabular}{|l||c|}\hline
     \textbf{材料}    &  \textbf{份量}\\ \hline\hline
    低筋面粉 & 100g \\ \hline
    无盐黄油 &  72g \\ \hline
    蛋黄 & 1 \\ \hline
    细砂糖 & 45g \\ \hline
    \end{tabular}
    \caption{材料表}
\end{table}

\begin{enumerate}
    \item 黄油切小块软化后,加入细砂糖打发至体积蓬松
    \item 加入蛋黄继续打发均匀
    \item 面粉过筛下黄油,用刮刀搅拌均匀成面团
    \item 将面团擀开,切成小的长方块,摆在烤盘上。饼干之间需留有一定空间
    \item 180度上小火中层烤14分钟
\end{enumerate}


\section{曲奇饼干}

\begin{table}[H]
    \centering
    \begin{tabular}{|l||c|}\hline
     \textbf{材料}    &  \textbf{份量}\\ \hline\hline
    低筋面粉 & 200g \\ \hline
    无盐黄油 &  160g \\ \hline
    牛奶 & 40g \\ \hline
    细砂糖 &80g \\ \hline
    盐 & 2g \\ \hline
    杏仁粉 & 40g \\ \hline
    \end{tabular}
    \caption{材料表}
\end{table}

\begin{enumerate}
    \item 牛奶加温倒入糖,搅均均匀至糖充分溶解
    \item 黄油室温放软到手指可以轻易戳洞的程度!电动打蛋器打发至变白,分次加入牛奶,打发至顺滑
    \item 黄油加入过筛好的面粉!和盐,杏仁粉,搅拌均匀成糊
    \item 曲奇放入裱花袋,挤在烤盘上。烤盘上注意需要垫一层烤纸
    \item 170度上下火中层烤15到20分钟
\end{enumerate}


\section{山药紫薯卷}

\begin{table}[H]
    \centering
    \begin{tabular}{|l||c|}\hline
     \textbf{材料}    &  \textbf{份量}\\ \hline\hline
    吐司 &  \\ \hline
    紫薯 &  \\ \hline
    山药 &  \\ \hline
    鸡蛋 &  \\ \hline
    牛奶 & \\ \hline
    \end{tabular}
    \caption{材料表}
\end{table}

\begin{enumerate}
    \item 紫薯、山药洗净蒸熟去皮,紫薯捣成泥,山药切小段
    \item 紫薯泥倒入适量牛奶,这样吃起来比较润
    \item 将鸡蛋的蛋液搅拌均匀,
    \item 切掉吐司的四边,用擀面杖把吐司擀成薄片,均匀的铺上紫薯泥,中间放上山药段,卷成卷。
    \item 紫薯卷外面涂抹一层蛋液,放入烤箱170度上下火烤10分钟至表面上色即可
\end{enumerate}


\section{长崎蛋糕(待修正)}

\begin{table}[H]
    \centering
    \begin{tabular}{|l||c|}\hline
     \textbf{材料}    &  \textbf{份量}\\ \hline\hline
    鸡蛋 & 4个 \\ \hline
    高筋面粉 & 100g \\ \hline
    细砂糖 &  110g \\ \hline
    蜂蜜 & 2小勺 \\ \hline
    热水 & 1勺半\\ \hline
    味霖 & 1勺\\ \hline
    \end{tabular}
    \caption{材料表}
\end{table}

\begin{enumerate}
    \item 鸡蛋恢复室温,蜂蜜和热水调成糖浆
    \item 把恢复到室温的鸡蛋打到盆里,分2-3次加入白糖打到6分发泡。
    \item 一点点加入蜂蜜和热水调成的糖浆,打到完全融合为止。最后加入味霖再轻微打一圈
    \item 分2-3次筛入面粉。用电打蛋器中速打匀。别怕消泡出筋,把面粉打匀是关键
    \item 把蛋糕液倒入模具里,颠一颠。用竹签子划Z字。这样可以消除多余的气体
    \item 180度烤10分钟后,蛋糕上色后盖上一层铝锡箔,烤箱降温到170度,烤40-45分钟
    \item 取出来后,立即盖上保鲜膜倒扣,尽量不要弄出皱褶。第二天再吃
\end{enumerate}



\section{红丝绒}

\begin{table}[h]
    \centering
    \begin{tabular}{|l||c|}\hline
     \textbf{材料}    &  \textbf{6寸份量}\\ \hline\hline
    \multicolumn{2}{|c|}{蛋糕体}\\ \hline
    低筋面粉     &  150 g \\ \hline
    无盐黄油     &  60g \\ \hline
    鸡蛋  & 1个  \\ \hline
    可可粉 &  10g  \\ \hline
    Butter Milk     &  120ml \\ \hline
    细砂糖     & 150g  \\ \hline
    红色素     & 适量  \\ \hline
    细盐     & 1/2茶匙  \\ \hline
    小苏打     & 1/2茶匙  \\ \hline
    白醋     & 1.5茶匙  \\ \hline
    香草精华     & 适量 \\ \hline
    \multicolumn{2}{|c|}{填充及抹面}\\ \hline
    无盐黄油     &  120g \\ \hline
    奶油奶酪     &  240g \\ \hline
    糖粉     &  120g \\ \hline
    \end{tabular}
    \caption{材料表}
\end{table}

\begin{enumerate}
    \item 黄油室温软化,鸡蛋拿出回温。
    \item 黄油加糖搅打顺滑,分次加入鸡蛋,继续搅打,使黄油和鸡蛋充分乳化。
    \item 加入红色素、香草精、白醋搅打均匀。红色素一点一点的加,加到自己满意的颜色为止。
    \item 分次加入Buttermilk,搅打均匀。此时的黄油已经是比较稀、可流动的状态了。
    \item 面粉、可可粉、盐、小苏打称量后过筛,加入黄油中,用刮刀拌匀。
    \item 如果太稠没有流动性会在烤后开裂,可以加微量牛奶。
    \item 180度上下火中层烤30到35分钟。
    \item 放凉后脱模
    \item 制作抹面和填充:
    \begin{enumerate}
    \item 奶油奶酪室温软化,搅打顺滑。
    \item 黄油室温软化,搅打顺滑
    \item 奶油奶酪和黄油混合,搅打顺滑。
    \item 糖粉过筛,加入到黄油奶酪糊中,先慢慢手动搅拌,润湿糖粉,以免飞溅。搅打均匀后,如果喜欢加点颜色,可以加点很淡的粉色,橘色,搅匀后就可以使用了。
    \end{enumerate}
\end{enumerate}

\section{抹茶戚风慕斯}

\begin{table}[h]
    \centering
    \begin{tabular}{|l||c|}\hline
     \textbf{材料}    &  \textbf{6寸份量}\\ \hline\hline
    \multicolumn{2}{|c|}{戚风(推荐多做一些)}\\ \hline
    鸡蛋     &  2个 \\ \hline
    低筋面粉     &  35g \\ \hline
    细砂糖  & 36g  \\ \hline
    色拉油 &  25g  \\ \hline
    牛奶     &  36g \\ \hline
    抹茶粉     & 4g  \\ \hline
    \multicolumn{2}{|c|}{慕斯液}\\ \hline
    淡奶油     &  200g \\ \hline
    细砂糖     &  15g \\ \hline
    牛奶     &  50g \\ \hline
    吉利丁片     &  10g \\ \hline
    抹茶粉     &  5g \\ \hline
    \multicolumn{2}{|c|}{糖酒液}\\ \hline
    水     &  30g \\ \hline
    细砂糖     &  10g \\ \hline
    朗姆酒     &  5g \\ \hline
    \end{tabular}
    \caption{材料表}
\end{table}

\begin{enumerate}
    \item 制作戚风:
    \begin{enumerate}
    \item 蛋黄和蛋白分开
    \item 蛋黄打散后加入牛奶和色拉油,搅打均匀。筛入低粉和抹茶粉,翻拌均匀。
    \item 蛋白加糖打发,打至干性发泡,提起打蛋头有短小直立的小尖角。
    \item 将1/3的蛋白与蛋黄糊翻拌均匀,再将蛋黄糊倒进蛋白盆中,继续翻拌均匀。
    \item 倒入模具,入烤箱,150度上下火,中层烤40分钟。
    \item 出炉倒扣在冷却架上冷却。等蛋糕完全冷却后再脱模
    \end{enumerate}
    \item 制作慕斯液:
    \begin{enumerate}
    \item 在淡奶油中加入糖,用电动打蛋器打出清晰的花纹即可。注意,这里虽然出现纹路了,但是淡奶油整体还是可以流动的状态
    \item 牛奶加热,放入已经用冷水泡软的吉利丁片,搅拌均匀。将牛奶吉利丁液保温,否则温度过低就会凝固
    \item 倒入奶油中拌匀,成为原味慕斯液。注意慕斯液的状态,慕斯液可以自己流平
    \end{enumerate}

    \item 将戚风上层不平整的部分削去。再分切成两片蛋糕片待用
    \item 制作糖酒液:将细砂糖倒入水中,加热至糖完全溶化,冷却后倒入朗姆酒
    \item 取出一片戚风,放入模具底部,刷上糖酒液后,导入一半的原味慕斯液,放入冰箱冷冻至稍微凝固。
    \item 剩下的慕斯液中筛入抹茶粉,搅拌均匀。抹茶粉可提前用10g热水(牛奶)将抹茶粉化开
    \item 冰箱中取出模具,在原味慕斯上放上第二片蛋糕片,并抹上糖酒液。最后倒入剩下的抹茶慕斯液
    \item 放入冰箱冷藏4小时以上直到凝固。脱模后在表面筛一层抹茶粉作为装饰
 
\end{enumerate}



\section{芒果流心慕斯}

\begin{table}[h]
    \centering
    \begin{tabular}{|l||c|}\hline
     \textbf{材料}    &  \textbf{6寸份量}\\ \hline\hline
    \multicolumn{2}{|c|}{慕斯)}\\ \hline
    芒果泥     &  350g \\ \hline
    吉利丁片     &  12.5g \\ \hline
    柠檬汁  & 15ml  \\ \hline
    动物性淡奶油 &  150g  \\ \hline
    细砂糖     &  60g \\ \hline
    芒果果肉     & 200g  \\ \hline
    \multicolumn{2}{|c|}{饼干底}\\ \hline
    无盐黄油     &  30g \\ \hline
    消化饼干     &  80g \\ \hline
    \multicolumn{2}{|c|}{镜面}\\ \hline
    水     &  50g \\ \hline
    吉利丁片     &  5g \\ \hline
    果泥     &  2勺 \\ \hline
    \end{tabular}
    \caption{材料表}
\end{table}

\begin{enumerate}
    \item 制作饼干底:
    \begin{enumerate}
    \item 无盐黄油放入热水中隔水融化
    \item 把消化饼用料理机打成粉末,或用擀面杖和保鲜袋来擀碎
    \item 把饼干碎与融化的黄油倒入碗中充分混匀
    \item 6寸蛋糕模具底层铺一张油纸,将饼干碎倒入模具,用刮刀压平压实成饼干底后,放入冰箱冷藏
    \end{enumerate}
    \item 制作慕斯液:
    \begin{enumerate}
    \item 芒果切丁,分成芒果泥以及夹心的部分,柠檬榨汁,吉利丁片泡水
    \item 将350g芒果肉,柠檬汁和细砂糖倒入料理机打成芒果泥
    \item 称出300g慕斯用的芒果泥,剩余用于流心。
    \item 吉利丁片沥水,隔水加热融化成液体。
    \item 把融化好的吉利丁液倒入慕斯部分的芒果泥里面用刮刀快速混匀(如果期间出现结块的现象,可以整碗芒果泥隔水加热一下,让凝固的吉利丁再次融化,如果吉利丁没有跟芒果泥混匀,会造成慕斯不成形,出水)
    \item 淡奶油打发至浓稠,出现纹路马上就消失的状态
    \item 将淡奶油倒入芒果泥中并用刮刀拌匀
    \item 把冰箱里面的饼干底拿出来,倒入三分一的慕斯糊,铺上一层芒果肉,再倒一层慕斯糊,外围一圈铺上芒果果肉中间倒入芒果流心部分的芒果泥。最后把剩余的所有慕斯糊倒入模具中,从边缘开始倒,以免中间的流心部分会溢出来,然后稍微晃动平整后,放入冰箱冷藏凝固2小时
    \end{enumerate}
    \item 制作镜面:取1片吉利丁片放入50g温水融化,倒入之前预留的2勺果泥中,搅拌均匀
    \item 取出冷藏好的慕斯蛋糕,表面倒上镜面液,放入冰箱冷藏过夜
\end{enumerate}



\section{芋圆烧仙草}

\begin{table}[H]
    \centering
    \begin{tabular}{|l||c|}\hline
     \textbf{材料}    &  \textbf{份量}\\ \hline\hline
    木薯粉 & 主食材重量的一半 \\ \hline
    红薯 &  \\ \hline
    紫薯 &  \\ \hline
    香芋 & \\ \hline
    仙草粉 & \\ \hline
    细砂糖 & \\ \hline
    \end{tabular}
    \caption{材料表}
\end{table}

\begin{enumerate}
    \item 将红薯,紫薯,香芋等蒸熟后去皮,加糖打成泥。
    \item 将薯泥分别加糖搅拌均匀,若湿度不够需要加热水。(紫薯偏干,需注意)
    \item 分别加入适量木薯粉揉成团。大约为薯泥的5到7成。步骤为关键,需要揉至整体均匀
    \item 搓成条,切成小块后表面洒上木薯粉。(如果不是当天吃,可以放入保鲜袋,冷冻1个月内都可食用)
    \item 煮芋圆需要先煮一锅沸水,将芋圆煮至浮起,捞出过凉水再放入碗中
    \item 仙草可以根据所购买的包装上的制作方法制作,若没有。则按以下方法:50g仙草粉加150克水调成无颗粒糊状。再将其倒入1升沸水中,小火不停搅拌,放凉固定后即可食用
    \item 芋圆烧仙草需配糖水汤底,推荐可食用红豆汤底
    \item 鲜芋仙的奶精可自制:淡奶欧加炼乳加热,搅拌均匀即可
\end{enumerate}

\section{Pancake}

\begin{table}[H]
    \centering
    \begin{tabular}{|l||c|}\hline
     \textbf{材料}    &  \textbf{份量}\\ \hline\hline
    低筋面粉 & 175g \\ \hline
    鸡蛋 & 55g 3个 \\ \hline
    牛奶 & 170ml \\ \hline
    白砂糖 & 30g\\ \hline
    盐 & 2g\\ \hline
    植物油 & 15g\\ \hline
    泡打粉 & 5g \\ \hline
    \end{tabular}
    \caption{材料表}
\end{table}

\begin{enumerate}
    \item 鸡蛋分蛋
    \item 蛋黄中倒入牛奶,盐和植物油,搅拌均匀。筛入低筋面粉和泡打粉用刮刀拌均
    \item 蛋白加入白砂糖,打发至湿性发泡
    \item 把蛋白分2-3次和入面糊中,切拌均匀
    \item 用不粘锅,小火热锅。取一个大汤勺,一次一勺,摊出来的pancake是10-12cm大。面糊要快点倒下去。
    \item 保持最小火,一直不翻面,直到面糊上面开始冒大气泡,并且小气泡已经破掉一小部分后即可翻面
\end{enumerate}


\section{冰糖葫芦}

\begin{table}[H]
    \centering
    \begin{tabular}{|l||c|}\hline
     \textbf{材料}    &  \textbf{份量}\\ \hline\hline
    山楂/金桔 &  \\ \hline
    冰糖/砂糖 &  \\ \hline
    \end{tabular}
    \caption{材料表}
\end{table}

\begin{enumerate}
    \item 如果用冰糖, 那么糖水比例为1:1,如果用砂糖, 糖水比例为2:1
    \item 使用能均匀导热的锅,下糖和水大火烧至沸腾并且糖融化
    \item 中火(150度)烧至糖水呈现黄色,并且有很多气泡
    \item 小火 熬至糖浆入冷水能马上变硬脆
    \item 熬制期间锅边上粘着的水滴要不断擦掉,否则会引起翻砂
    \item 用竹签将山楂串起,快速蘸上一圈糖浆后,用力甩到油纸上,可以使其快速凝固
\end{enumerate}


\section{红糖煎苹果}

\begin{table}[H]
    \centering
    \begin{tabular}{|l||c|}\hline
     \textbf{材料}    &  \textbf{份量}\\ \hline\hline
    苹果 &  \\ \hline
    红糖 &  \\ \hline
    黄油  &  \\ \hline
    \end{tabular}
    \caption{材料表}
\end{table}

\begin{enumerate}
    \item 苹果切片
    \item 热锅转小火下黄油融化
    \item 下苹果小火煎至软化
    \item 撒红糖翻滚,让其融化成糖浆即可出锅
\end{enumerate}


\section{覆盆子奶酪蛋糕}

\begin{table}[H]
    \centering
    \begin{tabular}{|l||c|}\hline
     \textbf{材料}    &  \textbf{8寸份量}\\ \hline\hline
    奶油奶酪 &  500g\\ \hline
    细砂糖 &  160g\\ \hline
    鸡蛋  &  3个\\ \hline
    面粉  &  40g \\ \hline
    柠檬屑  &  1个柠檬\\ \hline
    覆盆子  &  200g\\ \hline
    黄油  & 少量,仅用于擦模具防粘 \\ \hline
    \end{tabular}
    \caption{材料表}
\end{table}

\begin{enumerate}
    \item 将奶油奶酪和砂糖混合,打至蓬松发白
    \item 将鸡蛋打散。蛋液逐次倒入奶油奶酪和砂糖中,少量多次得加入鸡蛋液。一个是为了可以打入更多的空气,也是为了防止蛋奶分离。每次加入鸡蛋液后都要充分搅打均匀吸收后,才能再次加入
    \item 筛入面粉搅拌均匀
    \item 擦入柠檬屑
    \item 加入覆盆子。注意搅拌式不要将覆盆子捣碎
    \item 在模具内涂抹黄油防粘
    \item 将制作好的蛋糕糊倒入模具中
    \item 略微震模两下排除蛋糕糊中的空气
    \item 烤箱预热180度,烘烤35-40分钟直到蛋糕边缘凝固,中间还有些许的晃动感。
    \item 彻底冷却后放可脱模
\end{enumerate}




\section{栗子慕斯}

\begin{table}[H]
    \centering
    \begin{tabular}{|l||c|}\hline
     \textbf{材料}    &  \textbf{份量}\\ \hline\hline
    奶油奶酪 &  120g\\ \hline
    淡奶油 &  180g\\ \hline
    栗子茸  &  70g\\ \hline
    细砂糖  &  50g \\ \hline
    牛奶  & 80g\\ \hline
    朗姆酒  & 1小勺\\ \hline
    吉利丁片 & 5g \\ \hline
    巧克力酱 & 适量 \\ \hline
    糖粉 & 适量 \\ \hline
    栗子 & 适量 \\ \hline
    \end{tabular}
    \caption{材料表}
\end{table}

\begin{enumerate}
    \item 制作栗子蓉:
    \begin{enumerate}
    \item 剥去栗子的外壳,放水中煮至熟透,捞出来后碾成泥
    \item 将栗子泥放入炒锅里,开小火微微翻炒匀
    \item (可选)加入牛奶和糖一同翻炒使其液化,提高栗子浓度
    \end{enumerate}
    \item 将室温软化的奶油奶酪和细砂糖,搅拌打顺
    \item 加入牛奶,少量朗姆酒,和栗子蓉搅拌均匀
    \item 将冷水泡软的吉利丁片,隔热水融化之后加入到这个栗子糊中拌匀
    \item 待上面的栗子糊放置常温以后,直接倒入鲜奶油搅拌均匀,之后冰箱冷藏2-3小时左右凝固用
    \item 在栗子慕斯上挤上适量巧克力酱,之后装饰切小块的栗子碎,在顶部撒一些糖粉。(若选购的巧克力酱味道太浓,可以与牛奶调和,降低浓度)
\end{enumerate}


\section{苹果干}

\begin{table}[H]
    \centering
    \begin{tabular}{|l||c|}\hline
     \textbf{材料}    &  \textbf{份量}\\ \hline\hline
    苹果 &  \\ \hline
    柠檬水 &  \\ \hline
    菠萝汁 & \\ \hline
    \end{tabular}
    \caption{材料表}
\end{table}

\begin{enumerate}
    \item 苹果切片放进柠檬水+菠萝汁中浸泡
    \item 取出晾干
    \item 烤箱100度烤60分钟
\end{enumerate}



\section{栗子蒙布朗}

\begin{table}[h]
    \centering
    \begin{tabular}{|l||c|}\hline
     \textbf{材料}    &  \textbf{份量}\\ \hline\hline
    \multicolumn{2}{|c|}{蛋糕底)}\\ \hline
    低筋面粉 & 60g \\ \hline
    鸡蛋 & 4个 \\ \hline
    细砂糖 & 70g \\ \hline
    无盐黄油 & 30g \\ \hline
    \multicolumn{2}{|c|}{奶油馅}\\ \hline
    栗子泥     &  50g \\ \hline
    淡奶油     &  150g \\ \hline
    糖粉 &  15g \\ \hline
    细砂糖 & 适量 \\ \hline
    \multicolumn{2}{|c|}{栗子馅}\\ \hline
    栗子泥     &  180g \\ \hline
    淡奶油     &  适量 \\ \hline
    朗姆酒     &  20g \\ \hline
    无盐黄油  & 50g \\ \hline
    细砂糖 & 适量 \\ \hline
    \end{tabular}
    \caption{材料表}
\end{table}

\begin{enumerate}
    \item 制作蛋糕底:
    \begin{enumerate}
    \item 将鸡蛋的蛋清和蛋黄分开。在蛋清里加入细砂糖,用电动打蛋器打发至干性发泡
    \item 在另一个碗里打发蛋黄。将蛋黄打发到颜色变浅,体积蓬松的状态
    \item 将打发后的蛋黄倒入蛋清里,用刮刀从底部往上翻拌,使蛋黄和蛋清混合在一起
    \item 筛入低筋面粉至蛋糕液中,用刮刀进行翻拌,使面粉和蛋液完全混合,成为面糊
    \item 在面糊里倒入熔化成液态的黄油(黄油可以提前隔水加热或者用微波炉加热熔化成为液态),翻拌均匀
    \item 在铺了油纸的烤盘里倒入面糊,抹平。此做法不用蛋糕模具,如果准备用模具的话,可以减少原料,且注意蛋糕液在模具中的高度。蛋糕底不应过高。
    \item 上下火180度,烤13-15分钟,至蛋糕完全发起
    \item 待蛋糕冷却后,将油纸撕开,用直径4-6cm的圆形切模切出圆形的蛋糕片
    \end{enumerate}
 
    \item 制作奶油馅:
    \begin{enumerate}
    \item 若栗子泥制作中没有加糖,则需要加入糖来增加甜度
    \item 在栗子泥中加入少量淡奶油使栗子泥稀释并搅拌至栗子泥顺滑。若栗子泥本身水分充足,则可跳过这一步
    \item 将剩下的淡奶油加入糖粉,打发至干性发泡
    \item 将打发后的淡奶油与稀释后的栗子泥混合,搅拌均匀成奶油馅,放入冰箱冷藏备用。(不可放置过长时间,后续步骤应尽快)
    \end{enumerate}
    
    \item 制作栗子馅:
    \begin{enumerate}
    \item 若栗子泥制作中没有加糖,则需要加入糖来增加甜度
    \item 在栗子泥里加入软化后的黄油,用打蛋器搅打至均匀,将整块的栗子泥打至松发
    \item 加入朗姆酒,搅打均匀
    \item 在栗子泥里加入淡奶油,边加入边拌匀,调整浓稠程度到顺滑
    \item 装入裱花袋备用
    \end{enumerate}
    
    \item 在一片蛋糕片上挤上一圈奶油馅,放上第二片蛋糕片
    \item 再挤上一圈奶油馅
    \item 将栗子馅以线条的方式满满的挤在蛋糕外围,创造出蒙布朗的造型
    \item 在蛋糕表面撒少许可可粉(糖粉也可),并用水果或者一颗糖渍栗子放在表面作为装饰
    
\end{enumerate}


\section{轻奶酪蛋糕}

\begin{table}[H]
    \centering
    \begin{tabular}{|l||c|}\hline
     \textbf{材料}    &  \textbf{6寸份量}\\ \hline\hline
    奶油奶酪 & 100g \\ \hline
    无盐黄油 & 30g \\ \hline
    牛奶 & 60g\\ \hline
    淡奶油 & 20g \\ \hline
    玉米淀粉 & 20g \\ \hline
    鸡蛋 & 3个  \\ \hline
    细砂糖 & 45g \\ \hline
    柠檬汁 & 几滴 \\ \hline
    \end{tabular}
    \caption{材料表}
\end{table}

\begin{enumerate}
    \item 将奶油奶酪、黄油、30g牛奶、淡奶油放同一个打蛋盆里,隔水加热,使其融化
    \item 搅拌到黄油融化就可以离水,然后搅拌至没有奶油奶酪细颗粒,糊状基本细腻
    \item 将鸡蛋的蛋清和蛋黄分开,蛋黄放到奶油奶酪糊中
    \item 蛋黄搅拌均匀即可,不用过渡搅拌
    \item 玉米淀粉和30g牛奶混合成面粉糊
    \item 将蛋黄芝士糊和玉米淀粉糊混合,搅拌均匀,然后过筛备用
    \item 蛋清加几滴柠檬汁,分三次加入细砂糖,打至湿性发泡
    \item 蛋白霜和奶油奶酪糊分次加入,翻拌均匀,入模
    \item 用水浴法,上下火150度,放置下层,40分钟;然后调低温度120-130度,30分钟;最后调高温度160℃,5分钟等表面上色。
    \item 放凉后即可食用
\end{enumerate}


\section{重奶酪蛋糕}

\begin{table}[h]
    \centering
    \begin{tabular}{|l||c|}\hline
     \textbf{材料}    &  \textbf{7寸份量}\\ \hline\hline
    \multicolumn{2}{|c|}{蛋糕底)}\\ \hline
    低筋面粉 & 70g \\ \hline
    无盐黄油(发酵型) & 35g \\ \hline
    核桃 & 35g \\ \hline
    细砂糖 & 20g \\ \hline
    盐 & 少许 \\ \hline
    \multicolumn{2}{|c|}{奶酪蛋糕体}\\ \hline
    奶油奶酪(kiri)     &  330g \\ \hline
    细砂糖     &  100g \\ \hline
    酸奶油 &  145g \\ \hline
    无盐黄油(发酵型) & 37g \\ \hline
    香草荚 & 3分之1根 \\ \hline
    全蛋液 & 90g \\ \hline
    蛋黄 & 30g \\ \hline
    玉米淀粉 & 11g \\ \hline
    \end{tabular}
    \caption{材料表}
\end{table}

\begin{enumerate}
    \item 黄油放置室温软化,蛋液放置室温,香草荚刨开取出香草籽备用
    \item 制作蛋糕底:
    \begin{enumerate}
    \item 烤箱上下火170度中层烤6分钟至核桃薄皮有裂纹,并带有香味后,取出后用刷子刷掉表皮。若买来的核桃仁没有皮,只需将核桃烤熟即可
    \item 将低筋面粉,黄油,核桃,细砂糖和盐搅拌并用料理机粉碎到米粒状
    \item 将蛋糕底的颗粒倒入到模具中,压平压实。模具底需要用油纸铺垫
    \item 上下火160度,中层烤15到17分钟,至黄褐色取出。冷却后用脱模刀松动下蛋糕底与模具之间的空隙
    \end{enumerate}
 
    \item 制作蛋糕液:
    \begin{enumerate}
    \item 将奶油奶酪放入碗中打发至顺滑后,倒入香草籽和细砂糖继续用刮刀搅拌均匀,确保这一步后没有小颗粒,用打蛋器打发至顺滑。
    \item 将软化后的黄油倒入碗中搅拌均匀
    \item 倒入酸奶油搅拌均匀
    \item 将全蛋和蛋黄分别打散后分3次倒入,每加一次搅拌30秒,让材料裹入空气。(此步骤要求蛋液在室温,否则会油水分离)
    \item 搅拌均匀后应为液体状,此时筛入玉米淀粉,搅拌均匀后,用刮刀将盆壁上的面粉也一起拌匀
    \end{enumerate}
    \item 将蛋糕液倒入冷却的模具中,画Z震出气泡
    \item 水浴法:底层烤盘注入足够多的水并预热。 180度上下火中层烤30-35分钟。烤完后关闭烤箱,放置烤箱中40到60分钟直到冷却再取出,可避免塌陷。
    \item 于冰箱冷藏一天后享用
\end{enumerate}



\section{酸奶油}

\begin{table}[h]
    \centering
    \begin{tabular}{|l||c|}\hline
     \textbf{材料}    &  \textbf{7寸份量}\\ \hline\hline
    \multicolumn{2}{|c|}{做法一}\\ \hline
    淡奶油 & 100g \\ \hline
    柠檬汁 & 15g \\ \hline
    \multicolumn{2}{|c|}{做法二}\\ \hline
    淡奶油     &  100g \\ \hline
    酸奶     &  40g \\ \hline
    \multicolumn{2}{|c|}{做法三}\\ \hline
    淡奶油 & 250g \\ \hline
    酸奶菌粉 & 0.5 g \\ \hline
    \end{tabular}
    \caption{材料表}
\end{table}

\begin{enumerate}
    \item 所有做法都需要先将容器用沸水消毒并控干水分
    \item 做法1:
    \begin{enumerate}
    \item 柠檬汁和奶油倒入容器中,搅拌2分钟左右混合均匀,慢慢会出现凝结状态,盖上盖子,静置30分钟
    \item 放入冰箱冷藏一夜,即可使用。冷藏2天的柠檬酸奶油味道更好
    \end{enumerate}
 
    \item 做法2:
    \begin{enumerate}
    \item 将酸奶与奶油于容器中混合均匀。搅拌1分钟左右就能看到稍微凝结。常温静置8~12小时,可以直接用,也可以放入冰箱冷藏
    \end{enumerate}
    \item 做法3:
    \begin{enumerate}
    \item 奶油、菌粉倒入酸奶发酵机内, 发酵8小时左右,放凉冰箱冷藏保存即可。
    \item 使用之前用手动打蛋器搅拌细腻再使用
    \end{enumerate}
\end{enumerate}


\section{菠萝包}

\begin{table}[h]
    \centering
    \begin{tabular}{|l||c|}\hline
     \textbf{材料}    &  \textbf{14个菠萝包的份量}\\ \hline\hline
    \multicolumn{2}{|c|}{面团)}\\ \hline
    高筋面粉 & 500g \\ \hline
    细砂糖 & 55g \\ \hline
    奶粉 & 25g \\ \hline
    耐高糖酵母 & 5g \\ \hline
    全蛋液 & 50g \\ \hline
    牛奶 & 100g \\ \hline
    水 & 170g \\ \hline
    盐 & 5g \\ \hline
    无盐黄油 & 30g \\ \hline
    \multicolumn{2}{|c|}{酥皮}\\ \hline
    无盐黄油   &  100g \\ \hline
    糖粉(细砂糖) & 80g \\ \hline
    全蛋液 & 30g \\ \hline
    低筋面粉 & 140g \\ \hline
    奶粉 & 30g \\ \hline
    蛋黄 & 1个 \\ \hline
    \end{tabular}
    \caption{材料表}
\end{table}

\begin{enumerate}
    \item 制作面团:
    \begin{enumerate}
    \item 黄油放置室温软化
    \item 除黄油,盐以外,所有的面团材料混合并揉到光滑
    \item 加入盐和提前室温软化好的黄油块,揉至完全扩展阶段,可以拉出薄膜即可
    \item 把揉好的面团放容器内盖保鲜膜,进行基础发酵。直至面团发酵到两倍大时,手指沾面粉,轻戳一下,不回弹不回缩,发酵完成
    \item 取出发酵好的面团放揉面垫上,按揉几下,排出气体。称重面团,均匀分割成65g*14等份,分别滚圆。及时盖上保鲜膜松弛15分钟
    \end{enumerate}
    \item 制作酥皮:
    \begin{enumerate}
    \item 黄油放置室温软化
    \item 将黄油和糖粉搅拌均匀
    \item 一次性加入全蛋液搅拌均匀
    \item 筛入低筋面粉和奶粉后拌匀揉成面团
    \item 放入冰箱冷藏备用
    \end{enumerate}
    \item 将松弛好的面团,按扁,轻擀开。拍掉面团边缘的小气泡。搓成圆形,此时可以包上馅料。然后整成圆形。收口向下,盖上保鲜膜
    \item 从冰箱取出酥皮面团,搓成条,尽量搓均匀一致。可以在桌上洒上一层薄粉防粘。然后分割成26g*14等份
    \item 把酥皮面团搓圆,下面平铺一层保鲜膜,将酥皮面团放上去,上面再蒙上一层保鲜膜,用擀面杖在保鲜膜外面轻轻擀开,擀成厚薄均匀的面皮
    \item 把酥皮扣在面包体上面,慢慢把酥皮往下贴紧,压实后。拿掉保鲜膜,继续把面包体轻轻往酥皮里塞
    \item 整形搓圆
    \item 用刷子均匀扫一层蛋黄液在酥皮表面。用牙签或刮板在酥皮上轻轻划花纹。
    \item 划好花纹后,进行最后的发酵。二次发酵温度在37度左右,湿度75\%的环境下发酵至1.5倍大
    \item 烤箱180度上下火中层烤20分钟
\end{enumerate}


\section{枣糕}

\begin{table}[H]
    \centering
    \begin{tabular}{|l||c|}\hline
     \textbf{材料}    &  \textbf{份量}\\ \hline\hline
    鸡蛋 & 4个 \\ \hline
    去籽红枣 & 70g \\ \hline
    核桃肉 & 30g \\ \hline
    牛奶 & 50g \\ \hline
    朗姆酒 & 20g \\ \hline
    蜂蜜 & 20g \\ \hline
    低筋面粉 & 120g \\ \hline
    小苏打 & 2g \\ \hline
    红糖 & 60g \\ \hline
    植物油 & 50g \\ \hline
    盐(可选) & 0.5g \\ \hline
    \end{tabular}
    \caption{材料表}
\end{table}

\begin{enumerate}
    \item 红枣洗净,然后去籽切成红枣碎,把核桃敲成小碎粒和红枣混在一起,倒入牛奶和朗姆酒,搅拌均匀
    \item 倒入植物油搅拌均匀后备用
    \item 鸡蛋分蛋,红糖打碎打散
    \item 将10g红糖与蛋黄隔水搅拌均匀至蛋黄和红糖混合均匀
    \item 将蛋黄糊与红枣核桃混合,加入蜂蜜搅拌均匀备用
    \item 将50g红糖分三次加入蛋白中打发至干性发泡
    \item 取三分之一打好的蛋白霜放入红枣蛋黄糊里,快速翻拌均匀。再把剩下的蛋白倒入并翻拌均匀
    \item 筛入低筋面粉,小苏打和盐,快速翻拌均匀
    \item 倒入矮体模具中,烤箱上下火170度中层烤30分钟。若模具不是两根手指高的模具,需要增加烤箱时长
\end{enumerate}


\section{青柠戚风}

\begin{table}[H]
    \centering
    \begin{tabular}{|l||c|}\hline
     \textbf{材料}    &  \textbf{6寸份量}\\ \hline\hline
    鸡蛋 & 4个 \\ \hline
    细砂糖 & 70g \\ \hline
    青柠汁  & 25g \\ \hline
    青柠皮蓉 & 1个青柠 \\ \hline
    牛奶  & 35g \\ \hline
    低筋面粉  & 75g \\ \hline
    \end{tabular}
    \caption{材料表}
\end{table}

\begin{enumerate}
    \item 鸡蛋分蛋,青柠削皮切蓉
    \item 将蛋黄和35g细砂糖打散,加入牛奶,青柠汁和切碎的青柠皮蓉搅拌均匀。
    \item 筛入低筋面粉搅拌均匀备用
    \item 将35g糖分三次加入至蛋白,蛋白打发至干性发泡
    \item 将3分之1的蛋白霜加入蛋黄糊中,切拌拌匀。再倒回至剩下的蛋白霜中,切拌拌匀
    \item 倒入模具
    \item 上下火160度中层烤30分钟
    \item 冷却后即可食用
\end{enumerate}


\section{纽约青柠奶酪蛋糕}

\begin{table}[h!]
    \centering
    \begin{tabular}{|l||c|}\hline
     \textbf{材料}    &  \textbf{9寸份量}\\ \hline\hline
    \multicolumn{2}{|c|}{饼干底)}\\ \hline
    消化饼干 & 230g \\ \hline
    细砂糖  & 39g \\ \hline
    无盐黄油 & 113g \\ \hline
    \multicolumn{2}{|c|}{蛋糕体}\\ \hline
    细砂糖  &  236g \\ \hline
    玉米淀粉  & 8g \\ \hline
    奶油奶酪  & 227g \\ \hline
    L号鸡蛋  & 4个 \\ \hline
    酸奶油  & 160g \\ \hline
    重奶油  & 80ml \\ \hline
    青柠汁 & 120ml \\ \hline
    香草精 & 5g \\ \hline
    \multicolumn{2}{|c|}{装饰}\\ \hline
    重奶油 & 235ml \\ \hline
    细砂糖 & 33g \\ \hline
    青柠皮蓉(可选) & 适量 \\ \hline
    柠檬皮蓉(可选) & 适量 \\ \hline
    薄荷(可选) & 适量 \\ \hline
    蔓越莓(可选) & 适量 \\ \hline
    青柠片(可选)& 适量 \\ \hline
    \end{tabular}
    \caption{材料表}
\end{table}

\begin{enumerate}
    \item 重奶油的脂肪含量为36\%到40\%之间
    \item 准备模具,并在模具底部放置油纸。黄油放置室温软化
    \item 制作饼干底:
    \begin{enumerate}
    \item 将消化饼干和糖放入料理机中打碎成末
    \item 将饼干末放置碗中与软化后的黄油混合均匀
    \item 将混合物倒入模具并压紧压实
    \item 烤箱175度上下火 中层烤10分钟
    \item 取出冷却备用
    \end{enumerate}
    \item 制作蛋糕体:
    \begin{enumerate}
    \item 将奶油奶酪与糖和淀粉混合,打发至顺滑
    \item 逐个添加鸡蛋并搅拌均匀
    \item 倒入酸奶油和重奶油并混合均匀
    \item 倒入青柠汁和香草精并混合均匀
    \item 将气泡震出
    \item 水浴法:165度上下火中层烤60-65分钟
    \item 取出冷却1小时候,放入冰箱冷藏8小时
    \end{enumerate}
    \item 分三次添加糖,将重奶油打发至干性发泡。将打发后的奶油装入裱花袋备用
    \item 取出奶酪蛋糕,用裱花对奶酪蛋糕边缘裱上花,铺上皮蓉,薄荷。青柠片卡在奶油花之间,蔓越莓放在花上
\end{enumerate}


\section{青柠芝士慕斯}

\begin{table}[H]
    \centering
    \begin{tabular}{|l||c|}\hline
     \textbf{材料}    &  \textbf{6寸份量}\\ \hline\hline
    青柠戚风 & 1cm*2片 \\ \hline
    奶油奶酪 &  130g\\ \hline
    淡奶油 &  200g\\ \hline
    细砂糖  &  30g\\ \hline
    吉利丁片  &  5g \\ \hline
    酸奶  & 30g\\ \hline
    青柠汁  & 10g\\ \hline
    青柠皮屑 & 一个 \\ \hline
    蜂蜜 & 适量 \\ \hline
    镜面果胶 & 100g \\ \hline
    薄荷 & 适量 \\ \hline
    \end{tabular}
    \caption{材料表}
\end{table}

\begin{enumerate}
    \item 准备好所有原材料,吉利丁提前用冰水泡至15分钟以上,充分吸水变柔软。青柠用盐搓洗干净。奶油奶酪切丁放常温缓和变软。淡奶油冰箱冷藏。
    \item 青柠削皮屑,注意只取表面绿色的部分,不能包含白色脉络的部分,会造成苦的口感。切碎后加入少量蜂蜜腌制,更好的去掉苦味。
    \item 泡好的吉利丁取出放入酸奶小碗里,放进微波炉里叮数秒,至吉利丁完全融化。混合酸奶和吉利丁。(如果此时混合物偏热,需降温至35度)
    \item 奶油奶酪加细砂糖,打发至顺滑
    \item 加入4分之3腌制好的柠檬皮屑入芝士糊,再加入10g青柠汁
    \item 加入酸奶与吉利丁的混合物,用打蛋器搅拌均匀
    \item 淡奶油单独打发至出现纹路即可
    \item 混合淡奶油和芝士面糊,切拌均匀成慕斯糊
    \item 取出模具,将青柠戚风放一片垫底,倒入一半慕斯糊并用刮刀抹平。再放入第二片戚风片后,倒入剩余的慕斯糊,用刮刀铺平。放入冰箱冷冻1小时以上
    \item 取剩余的柠檬皮屑,混合薄荷碎和镜面果胶
    \item 取出冷冻好的慕斯蛋糕,将镜面液体倒在慕斯上,放入冰箱冷藏6小时后享用
\end{enumerate}


\section{青柠慕斯}

\begin{table}[H]
    \centering
    \begin{tabular}{|l||c|}\hline
     \textbf{材料}    &  \textbf{6寸份量}\\ \hline\hline
    青柠戚风 & 1cm*2片 \\ \hline
    奶油奶酪 &  100g\\ \hline
    淡奶油 &  150g\\ \hline
    牛奶 & 70g \\ \hline
    细砂糖  &  适量\\ \hline
    吉利丁片  &  11g \\ \hline
    青柠汁  & 60g\\ \hline
    青柠皮屑 & 一个 \\ \hline
    蜂蜜 & 60g \\ \hline
    镜面果胶 & 100g \\ \hline
    薄荷 & 适量 \\ \hline
    \end{tabular}
    \caption{材料表}
\end{table}

\begin{enumerate}
    \item 准备好所有原材料,吉利丁提前用冰水泡至15分钟以上,充分吸水变柔软。青柠用盐搓洗干净。奶油奶酪切丁放常温缓和变软。淡奶油冰箱冷藏。
    \item 青柠削皮屑,注意只取表面绿色的部分,不能包含白色脉络的部分,会造成苦的口感。切碎后加入蜂蜜腌制,更好的去掉苦味。
    \item 奶油奶酪加细砂糖和牛奶,打发至顺滑
    \item 加入4分之3腌制好的柠檬皮屑入芝士糊,再加入45g青柠汁
    \item 然后将泡软的吉利丁片去掉多余的冰水,隔温水搅拌至溶解后,把4分之3的吉利丁溶液加入到芝士糊中
    \item 淡奶油单独打发至出现纹路即可
    \item 混合淡奶油和芝士面糊,切拌均匀成慕斯糊
    \item 取出模具,将青柠戚风放一片垫底,倒入一半慕斯糊并用刮刀抹平。再放入第二片戚风片后,倒入剩余的慕斯糊,用刮刀铺平。放入冰箱冷冻1小时以上
    \item 取剩余的柠檬皮屑,混合薄荷碎,青柠汁和镜面果胶
    \item 取出冷冻好的慕斯蛋糕,将镜面液体倒在慕斯上,放入冰箱冷藏6小时后享用
\end{enumerate}


\section{半熟蜂蜜蛋糕}

\begin{table}[H]
    \centering
    \begin{tabular}{|l||c|}\hline
     \textbf{材料}    &  \textbf{6寸份量}\\ \hline\hline
    全蛋 & 1个 \\ \hline
    蛋黄 & 3个 \\ \hline
    细砂糖 & 15g \\ \hline
    蜂蜜 & 15g \\ \hline
    低筋面粉 & 20g \\ \hline
    香草精(可选) & 两三滴 \\ \hline
    \end{tabular}
    \caption{材料表}
\end{table}

\begin{enumerate}
    \item 鸡蛋和蛋黄加入细砂糖(也可以用30g糖粉),在40度的热水中隔水打发至顺滑
    \item 倒入蜂蜜和香草精(可选,如果鸡蛋腥味重)再搅拌均匀
    \item 筛入面粉,用刮刀搅拌均匀至无颗粒
    \item 170度上下火中层烤10到13分钟
    \item 烤完后直接拿出放凉,待回室温并凹陷后即可食用
\end{enumerate}


\section{佛卡夏面包}

\begin{table}[H]
    \centering
    \begin{tabular}{|l||c|}\hline
     \textbf{材料}    & $560cm^2$ \textbf{份量}\\ \hline\hline
    面粉 & 500g \\ \hline
    小麦粉 & 20g \\ \hline
    干酵母 & 15g \\ \hline
    盐 & 适量 \\ \hline
    橄榄油 & 50ml \\ \hline
    温热水 & 320ml \\ \hline
    胡椒 & 适量 \\ \hline
    橄榄 & 适量 \\ \hline
    干番茄 & 适量 \\ \hline
    迷迭香 & 大量 \\ \hline
    \end{tabular}
    \caption{材料表}
\end{table}
\begin{enumerate}
    \item 在碗里放入面粉, 小麦粉,干酵母和少量盐调味
    \item 将橄榄油与温热水混合
    \item 将油水逐步倒入碗中混合
    \item 揉成团后让它舒展10分钟后开始放在案板上用拉翻的手法使其表面光滑,不要揉出面筋。
    \item 放回碗中,置温暖地方发面30-60分钟。
    \item 在模具底部撒盐。搁入发至两倍大的面团。用手蘸上橄榄油将面团按至边缘,同时按出一个个小洞。将橄榄,干番茄放入洞中。
    \item 撒盐和胡椒在表面调味。撒上大量新鲜迷迭香,并倒一点橄榄油在上方
    \item 200度上下火中层烤30分钟
\end{enumerate}

这个配方可以做560平方厘米的佛卡夏面包。注意模具面积。烤后的高度约为烤前的两到三倍

\section{披萨}

\begin{table}[H]
    \centering
    \begin{tabular}{|l||c|}\hline
     \textbf{材料}    &  \textbf{4个披萨的份量}\\ \hline\hline
    高筋面粉 & 500g \\ \hline
    盐 & 15g \\ \hline
    橄榄油 & 60ml \\ \hline
    干酵母 & 15g \\ \hline
    糖 & 15g \\ \hline
    温热水 & 325g \\ \hline
    披萨馅料 & \\ \hline
    \end{tabular}
    \caption{材料表}
\end{table}

\begin{enumerate}
    \item 将高筋面粉和盐倒入碗中
    \item 将温热水,橄榄油,干酵母和糖混合均匀
    \item 将油水倒入碗中和成面团
    \item 松弛几分钟后和成光滑面团。醒面1小时发至两倍大
    \item 排气,分四等分,揉至披萨形状
    \item 热锅,热油,中火。下披萨面团。等到中间起泡时,可以涂番茄酱,奶酪等材料。
    \item 烤箱上下火220度烤至上方出现梅拉德反应时结束。(尽可能将烤箱用最高温进行烤制。上一步耗时极短,且需要连贯,烤箱需要提前预热)
\end{enumerate}


\section{法棍}

\begin{table}[H]
    \centering
    \begin{tabular}{|l||c|}\hline
     \textbf{材料}    &  \textbf{4个法棍份量}\\ \hline\hline
    中筋/高筋面粉 & 450g \\ \hline
    盐 & 8g \\ \hline
    干酵母 & 半勺 \\ \hline
    室温水 & 340g \\ \hline
    \end{tabular}
    \caption{材料表}
\end{table}

\begin{enumerate}
    \item 酵母用水化开,加入面粉和盐,搅拌至看不到干粉
    \item 室温发酵1小时。分别在30分钟,60分钟的时候拉伸折叠法面团
    \item 盖上保鲜纸,冰箱隔夜发酵
    \item 第二天,分割,滚圆。回温40分钟
    \item 整形,最后发酵40分钟。室温高于21摄氏度的话,要缩短发酵时间
    \item 烤箱下层准备小石头,285度预热,将石头也烤热
    \item 割包
    \item 将面团放入烤箱中层,并给下层小石子倒入开水出蒸汽。285度烤10到12分钟或240度烤20分钟
\end{enumerate}

\section{奶香馒头}

\begin{table}[H]
    \centering
    \begin{tabular}{|l||c|}\hline
     \textbf{材料}    &  \textbf{份量}\\ \hline\hline
    中筋面粉 & 200g \\ \hline
    糖 & 15g \\ \hline
    干酵母 & 2g \\ \hline
    温热水 & 20ml \\ \hline
    牛奶 & 100g \\ \hline
    植物油 & 15g \\ \hline
    \end{tabular}
    \caption{材料表}
\end{table}

\begin{enumerate}
    \item 将糖和酵母倒入温热水中,搅拌均匀
    \item 将酵母水,牛奶和油倒入中筋面粉中并揉成团
    \item 醒5分钟后揉至光滑
    \item 再醒10分钟。整形切块后放入蒸锅上醒面。醒到1点5倍大
    \item 不用取出,直接开火蒸。蒸10分钟即可
\end{enumerate}