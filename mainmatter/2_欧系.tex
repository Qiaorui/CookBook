\chapter{西餐}

\section{油炸小鱿鱼}

\begin{table}[H]
    \centering
    \begin{tabular}{|l||c|}\hline
     \textbf{材料}    &  \textbf{份量}\\ \hline\hline
    鱿鱼 & 1000g \\ \hline
    面粉 &  300g \\ \hline
    鸡蛋 & 2  \\ \hline
    啤酒 & 300ml  \\ \hline
    水 & 适量 \\ \hline
    柠檬 & 1颗 \\ \hline
    盐 & 适量 \\ \hline
    胡椒 & 适量 \\ \hline
    \end{tabular}
    \caption{材料表}
\end{table}

\begin{enumerate}
    \item 小鱿鱼去骨处理干净
    \item 将面粉,鸡蛋,啤酒和水调整面粉糊,加入少量盐和胡椒做底味
    \item 烧一锅热油,鱿鱼挂糊后炸至金黄。需复炸一次
    \item 炸完后滤油摆盘,吃前可以挤柠檬汁到鱿鱼上,撒上盐和胡椒
\end{enumerate}


\section{蒜烤面包}

\begin{table}[H]
    \centering
    \begin{tabular}{|l||c|}\hline
     \textbf{材料}    &  \textbf{份量}\\ \hline\hline
    蒜 &  \\ \hline
    面包 &   \\ \hline
    牛至 & 一勺  \\ \hline
    橄榄油 &   \\ \hline
    胡椒 & 适量 \\ \hline
    盐 & 1颗 \\ \hline
    \end{tabular}
    \caption{材料表}
\end{table}

\begin{enumerate}
    \item 面包切片涂少量橄榄油进烤箱180度烤7到10分钟
    \item 将橄榄油,蒜末,牛至,盐和胡椒调成烤酱
    \item 涂酱至面包上后继续低温烤5分钟
\end{enumerate}


\section{土豆鸡蛋饼}

\begin{table}[H]
    \centering
    \begin{tabular}{|l||c|}\hline
     \textbf{材料}    &  \textbf{份量}\\ \hline\hline
    土豆 &  3-4个\\ \hline
    洋葱 &  1个 \\ \hline
    鸡蛋 & 4-5个  \\ \hline
    油 &   \\ \hline
    盐 &  \\ \hline
    \end{tabular}
    \caption{材料表}
\end{table}

\begin{enumerate}
    \item 把土豆切成片,不需要太薄,大概3毫米左右即可,洋葱切成小丝备用
    \item 热锅热大量油中火,等油至五分热时放入土豆,视个人口味加盐,加2-3小勺就好,待土豆变色后加入洋葱。用最小火熬至土豆可以轻易煸碎。中间需要不时翻炒,以免糊锅。
    \item 在等待期间,把四个鸡蛋打入一个比较大的碗,放入少许盐打匀备用。
    \item 用铲子把土豆捣成小碎块,熄火,用滤网把油沥出,放入小碗备用。已经碎成泥的土豆及洋葱倒入鸡蛋液内,把土豆与鸡蛋充分搅匀,不需要把土豆完全搅成泥,留有块状口感会更好。
    \item 热锅热多油中火,将之前沥出的油倒一些进平底锅,油热后倒入蛋液。同时不时晃动锅子确保不沾。慢火煎五-十分钟。等底层基本凝固后熄火,可以用铲子铲入底部确认煎饼没有沾在锅底。注意此步骤为避免焦底,注意控制油量
    \item 找一个大于等于锅子尺寸的盘子盖在平底锅上,一手持锅柄一手托盘,然后翻盘。此时上层的蛋液仍处于液体状所以动作要快。
    \item 热锅热油,将半熟的煎饼划入锅中,用铲子修正一下边缘形状,慢火煎五-十分钟,用刀插入煎饼来确认是否凝固来确认成熟。或者用牙签插入,若无流动的蛋液即为成熟。
\end{enumerate}

\section{猪排配酸甜灯笼椒}

\begin{table}[H]
    \centering
    \begin{tabular}{|l||c|}\hline
     \textbf{材料}    &  \textbf{份量}\\ \hline\hline
    红皮洋葱 & 半个 \\ \hline
    红灯笼椒 & 2个  \\ \hline
    红葡萄酒醋 & 3茶匙  \\ \hline
    带骨厚猪排 &  200g x 2 \\ \hline
    橄榄油 &  \\ \hline
    蒜 & 2瓣 \\ \hline
    百里香 & 一小把 \\ \hline
    黄油 &  适量 \\ \hline
    糖 & 1茶匙 \\ \hline
    罗勒叶 & 1把 \\ \hline
    盐 &  \\ \hline
    胡椒 &  \\ \hline
    \end{tabular}
    \caption{材料表}
\end{table}

\begin{enumerate}
    \item 灯笼椒切丝,洋葱切丝,罗勒切碎, 大蒜拍扁
    \item 用刀将肉皮切断,大概切5毫米的深度,每隔3-4厘米切一下肉皮。这样做是为了防止肉在煎的过程中卷曲。切记不要切刀肉的部分,只切肉皮
    \item 用盐和胡椒腌制猪肉,用手拍一拍猪肉使盐和胡椒的味道能尽量进入猪肉里面
    \item 热锅热油中火,下洋葱灯笼椒炒香。加入盐和胡椒调味,再加入砂糖翻炒4到5分钟直到变软变色
    \item 锅中加入葡萄酒醋,炒1分钟待红椒煸软后转小火加入2茶匙的初榨橄榄油,小火炖煮2到3分钟。
    \item 锅中加入切碎的罗勒叶继续小火煮30秒,出香味后盛出备用
    \item 热锅热油,下腌制好的猪排和大蒜煎香。加入百里香,煎2-3分钟直到猪排变色,然后给猪排翻面,另一面煎2-3分钟。可将百里香垫到猪排下面,把大蒜铲碎一点儿它的味道充分释放
    \item 猪排快完全成熟的时候,加入三块黄油,等黄油溶化后,用勺子将锅内的汤汁/黄油,不断浇到猪排上使猪排富有汁水。可以将猪排立起来,使猪皮接触锅底,让肉内的脂肪流出来一些。将锅内大蒜的皮取出,将锅内的大蒜和百里香铺到猪排表面。之后将煎好的猪排盛出,放置5-10分钟,再用锅中的油在盘子里浇淋一遍猪排
    \item 将准备好的红椒洋葱丝摆盘,摆上建好的猪排,淋一些炒好的洋葱红椒的汁水即可。
\end{enumerate}


\section{烤肋排}

\begin{table}[H]
    \centering
    \begin{tabular}{|l||c|}\hline
     \textbf{材料}    &  \textbf{份量}\\ \hline\hline
    猪肋排 & 6根 \\ \hline
    小土豆 & 若干  \\ \hline
    Hunt's 烤肉酱 & 100g  \\ \hline
    蒜 &  3瓣 \\ \hline
    橄榄油 & 1勺 \\ \hline
    蜂蜜 & 2勺 \\ \hline
    百里香 & 适量 \\ \hline
    黑胡椒 &  适量 \\ \hline
    辣椒碎 & 适量 \\ \hline
    \end{tabular}
    \caption{材料表}
\end{table}

\begin{enumerate}
    \item 蒜切粒,土豆洗净切滚刀块
    \item 猪肋排洗净滤水后,均匀涂抹烤肉酱和蒜粒,撒上黑胡椒按摩,放入保鲜袋包好放进冰箱冷藏过夜。(烧烤酱也可以按60\%的烧烤酱和40\%的番茄酱混合)
    \item 土豆混合橄榄油、黑胡椒、辣椒碎、百里香,放入保鲜袋冷藏过夜
    \item 去除肋排上的蒜粒,加入蜂蜜和剩余酱汁按摩均匀后放置烤盘上。小土豆去除表面香料,放置烤盘上。(如果土豆过大,可以将其切半)
    \item 烤箱预热180度。中层180度上小火烤30分钟。需要下层铺锡纸接油
    \item 肋排翻面再烤30分钟。此时烤盘上的锡纸应该已经被酱料烤糊,需换一张
    \item 出烤箱后撒黑胡椒
\end{enumerate}


\section{柠檬茴香百里香盐焗鱼}

\begin{table}[H]
    \centering
    \begin{tabular}{|l||c|}\hline
     \textbf{材料}    &  \textbf{份量}\\ \hline\hline
    鲈鱼 & 500g一条 \\ \hline
    粗盐 & 900g  \\ \hline
    柠檬 & 1颗  \\ \hline
    百里香 &  1到3束 \\ \hline
    茴香籽 & 2tbsp \\ \hline
    蛋清 & 1个 \\ \hline
    白酒 & 1tbsp \\ \hline
    橄榄油 &  少量 \\ \hline
    \end{tabular}
    \caption{材料表}
\end{table}

\begin{enumerate}
    \item 海鱼洗净处理干净后擦干,柠檬擦屑,百里香取茎留叶,小茴香切碎。
    \item 混合百里香叶、小茴香碎、柠檬屑,倒入1大勺橄榄油,拌均匀
    \item 将上一步的混合物抹在鱼身内外,放入冰箱腌制半小时
    \item 起一个大碗,放入一个蛋的蛋清,与1勺白酒(或料酒)混合均匀后倒入所有粗盐,继续拌均匀
    \item 烤盘垫锡纸,先在锡纸上薄薄铺一层与蛋清拌均匀的盐,把鱼放上去后,再在鱼表面覆盖一层厚厚的盐,密封严实
    \item 烤箱205度预热5分钟后,200度上下管置于烤箱中层烤30分钟
    \item 取出后,用刀背敲开盐层,在鱼上淋少许橄榄油,挤柠檬汁一起淋在鱼肉上,或直接整盘上菜,或取出全鱼,配少许蔬菜装盘食用
\end{enumerate}

注意:香料不能过多,味道会太重,特别是百里香。鱼鳃处由于被移除了,所以空着,如果用盐塞满会容易过咸,可以尝试用柠檬代替

\section{虾仁意面}

\begin{table}[H]
    \centering
    \begin{tabular}{|l||c|}\hline
     \textbf{材料}    &  \textbf{份量}\\ \hline\hline
    番茄 & 两个 \\ \hline
    虾 & 7只  \\ \hline
    意面 & 1人份  \\ \hline
    蒜 &  1瓣 \\ \hline
    橄榄油 & 少量\\ \hline
    盐 & 少量 \\ \hline
    黑椒 &  少量 \\ \hline
    欧芹碎 &  少量 \\ \hline
    \end{tabular}
    \caption{材料表}
\end{table}

\begin{enumerate}
    \item 番茄焯水去皮切块,虾仁开线焯熟,蒜切碎
    \item 烧一锅水,煮沸后加入少许盐和橄榄油,下意面煮熟。煮熟后捞出备用
    \item 热锅热油小火,炒香蒜末后下番茄。转中火边炒边用锅铲压番茄,把汁水压出来。然后下虾仁,翻炒一阵后,下意面
    \item 出锅前撒上盐、黑椒椒和欧芹碎调味即可
\end{enumerate}

\section{土豆泥}

\begin{table}[H]
    \centering
    \begin{tabular}{|l||c|}\hline
     \textbf{材料}    &  \textbf{份量}\\ \hline\hline
    夏洛特土豆 & 500g \\ \hline
    全脂牛奶 & 100ml  \\ \hline
    无盐黄油 & 150g  \\ \hline
    细海盐 &  10g \\ \hline
    白胡椒粉(可选) & 少量\\ \hline
    \end{tabular}
    \caption{材料表}
\end{table}

\begin{enumerate}
    \item 将土豆洗净,放入锅中,加入大量的冷水,将水煮至沸腾后,中火续煮30分钟左右,用竹签可以轻易插入。(土豆必须带皮煮,让皮阻隔水分;以冷水煮土豆,让温度可以缓缓进入内部,避免内外烹饪程度差异过大)
    \item 将土豆去皮,放入最细的研磨网中压出细腻的土豆泥。(不可用搅拌机,打出的土豆泥会成浆糊状,口感差)
    \item 把压好的土豆泥放入不粘锅中,小火搅拌加热5分钟。(蒸发掉部分水分,使味道更加浓郁醇厚)
    \item 加热土豆泥时小火煮沸全脂牛奶。
    \item 加入足够冰凉的黄油块至搅拌土豆的锅中。(足够冰凉的黄油会减缓融化时间,使黄油混合更加均匀,土豆泥更具粘着性)。 搅拌至黄油完全被土豆泥吸收
    \item 加入煮好的牛奶,不断搅拌,直到牛奶完全被土豆泥吸收。用盐和白胡椒粉调味。(不可黑胡椒粉,看起来会很像杂质)
    \item 将土豆泥装入碗中或者挤成形状,把之前预留的一小块黄油融化,刷在土豆泥表面,会形成隔绝空气的薄层,防止土豆泥变干,并会使表层香味浓郁)
\end{enumerate}


\section{法式南瓜奶油浓汤}

\begin{table}[H]
    \centering
    \begin{tabular}{|l||c|}\hline
     \textbf{材料}    &  \textbf{份量}\\ \hline\hline
    南瓜 &  \\ \hline
    生姜 & 适量 \\ \hline
    蒜(可选) & 少量 \\ \hline
    洋葱 &  半个 \\ \hline
    胡萝卜 & 少量\\ \hline
    黄油 & 少量\\ \hline
    淡奶油 & 适量\\ \hline
    盐 & 适量\\ \hline
    糖 & 少量\\ \hline
    \end{tabular}
    \caption{材料表}
\end{table}

\begin{enumerate}
    \item 南瓜去皮去籽切片,蒜剥皮,姜切片,洋葱切碎,胡萝卜切片。
    \item 用一口煮锅,放黄油烧热,融化后小火炒洋葱。炒出香甜味后,倒入蒜,生姜,胡萝卜。
烧至香味出来后,加入南瓜。
    \item 加水至比南瓜水平线低一点,大火煮开后中火煮20分钟左右,等南瓜软化后加盐,胡椒和糖调味。
    \item 如果有料理机的话,直接把材料倒入料理机,打成浓汤即可。
    \item 如果没有的话,开小火继续慢煮,等南瓜彻底软烂,然后碾碎。
    \item 取筛子过滤掉杂质和固体,将过滤后的浓汤重新放入锅中开小火
    \item 加入淡奶油调味,烧至稠度适中,最后轻轻搅拌即可盛出。(这一步若浓汤太稠,可以适当加水。淡奶油放到最后再进行调味是为了防止过早沸腾)
\end{enumerate}


\section{蜂蜜鸡胸肉}

\begin{table}[H]
    \centering
    \begin{tabular}{|l||c|}\hline
     \textbf{材料}    &  \textbf{份量}\\ \hline\hline
    鸡胸肉 &  \\ \hline
    盐 &  \\ \hline
    蜂蜜 &  \\ \hline
    白胡椒 &  \\ \hline
    黑胡椒 & \\ \hline
    蒜 & \\ \hline
    橄榄油 & \\ \hline
    \end{tabular}
    \caption{材料表}
\end{table}

\begin{enumerate}
    \item 蒜切末,将橄榄油,蜂蜜和蒜末混合均匀成酱。(蜂蜜不可太多,会导致焦锅)
    \item 取黑白胡椒,盐均匀涂抹鸡胸肉
    \item 再外面裹一层腌酱,放置冰箱腌制过夜
    \item 去除鸡胸肉表面的蒜粒,封一层橄榄油
    \item 热锅无油小火,盖上锅盖,煎至正反面梅拉德反应即可出锅。
\end{enumerate}


\section{北非蛋}

\begin{table}[H]
    \centering
    \begin{tabular}{|l||c|}\hline
     \textbf{材料}    &  \textbf{份量}\\ \hline\hline
    青红甜椒 &  \\ \hline
    洋葱 &  \\ \hline
    孜然粒 &  \\ \hline
    橄榄油 & 适量 \\ \hline
    蒜 & \\ \hline
    辣椒 & \\ \hline
    西红柿 & \\ \hline
    盐 & \\ \hline
    胡椒 & \\ \hline
    鸡蛋 & \\ \hline
    小葱 & \\ \hline
    香菜 & \\ \hline
    \end{tabular}
    \caption{材料表}
\end{table}

\begin{enumerate}
    \item 洋葱切碎,青红椒切小丁,大蒜切片,辣椒切碎,西红柿去皮切碎,小葱和香菜切碎
    \item 热锅热多油小火,将洋葱碎和青红椒放入锅中翻炒。(这一步橄榄油可以多放一点,后续西红柿的水份会与油产生反应蒸发掉较多油,所以为防止烧焦,适当多一点油)
    \item 出香味后,下大蒜,辣椒,放入锅中同洋葱和青红椒一起翻炒,至食材变软
    \item 将西红柿放入锅中,继续翻炒至出水
    \item 加入孜然籽,盐和胡椒调味,直至锅中食材变得软烂。之后挖出四个圈,用来打鸡蛋
    \item 将鸡蛋打入圈中。盖上锅盖,开小火,将鸡蛋煎制5分钟
    \item 将青葱和香菜切碎,出锅前撒在锅内即可
\end{enumerate}



\section{意式海鲜汤}

\begin{table}[H]
    \centering
    \begin{tabular}{|l||c|}\hline
     \textbf{材料}    &  \textbf{份量}\\ \hline\hline
    海虹 & 6枚 \\ \hline
    大虾  &  8只 \\ \hline
    扇贝  &  6只\\ \hline
    圣女果  & 80g \\ \hline
    洋葱  &  4分之1\\ \hline
    蒜  & 4瓣 \\ \hline
    白葡萄酒 & 20ml \\ \hline
    小米椒 & 2只 \\ \hline
    罗勒叶  & 1根 \\ \hline
    橄榄油  & 20ml \\ \hline
    西红柿酱  & 125ml \\ \hline
    海盐  & 适量 \\ \hline
    白胡椒  & 适量 \\ \hline
    \end{tabular}
    \caption{材料表}
\end{table}

\begin{enumerate}
    \item 海鲜洗净处理后滤水,洋葱切丝,罗勒切碎
    \item 热锅热油小火,下洋葱,大蒜炒香后加盐和小米椒
    \item 转大火下虾,扇贝和圣女果翻炒。直到虾变色,扇贝表面梅拉德反应
    \item 倒白葡萄酒,下海虹后,加400ML热水和西红柿酱。搅拌均匀
    \item 收汁到一半时,撒入罗勒碎出锅即可
\end{enumerate}




\section{烤章鱼(待修正)}

\begin{table}[H]
    \centering
    \begin{tabular}{|l||c|}\hline
     \textbf{材料}    &  \textbf{份量}\\ \hline\hline
    章鱼 &  \\ \hline
    各色香料  &   \\ \hline
    盐   &   \\ \hline

    \end{tabular}
    \caption{材料表}
\end{table}

\begin{enumerate}
    \item 一锅热水放入香料,下章鱼三起三方,煮30到40分钟
    \item 煮熟后取出切块再进行烤制
    \item 烤完后撒上盐,甜椒粉和胡椒粉即可
\end{enumerate}