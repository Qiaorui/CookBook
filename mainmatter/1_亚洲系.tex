\chapter{亚洲餐}

\section{红烧肉/排骨}

\subsection{传统做法}

\begin{table}[h]
    \centering
    \begin{tabular}{|l||c|}\hline
     \textbf{材料}    &  \textbf{份量}\\ \hline\hline
    五花肉/排骨     &  600 g \\ \hline
    糖(最好是冰糖)     &   \\ \hline
    盐     &    \\ \hline
    酱油     &   \\ \hline
    黄酒     &   \\ \hline
    大葱     &  一整颗 \\ \hline
    生姜     &  四五片 \\ \hline
    八角     &  三四颗 \\ \hline
    桂皮     &  一段\\ \hline
    香叶     &  两片 \\ \hline
    花椒     &  数颗 \\ \hline
    小葱(装饰)     &   \\ \hline
    \end{tabular}
    \caption{材料表}
\end{table}

\begin{enumerate}
    \item 整块肉冷水下锅加两勺黄酒焯水,去腥去血沫,方便定型。想要皮脆的话,也可改用炸的方式。
    \begin{itemize}
        \item 等焯水的同时准备香料,生姜切片,大葱切段。
    \end{itemize}
    \item 焯好后取出用热水洗干净,切块备用。不可用凉水,会导致肉收缩回生,导致烹调时间变长。
    \item 中小火倒少油入锅,倒适量冰糖(最少4快)炒糖色。推荐用水油炒法,冒小泡后即可倒入肉块翻炒。或者用油炒糖油,这样的话,最后步骤就不用加糖。
    \item 上色均匀后倒入香料(大葱,姜,八角,桂皮,香叶,花椒)煸炒。
    \item 出香味后倒入适量酱油和黄酒,这个步骤黄酒可以多一点甚至用黄酒代替水也没事,然后开大火倒入热水直到没过肉。可以撒一点白胡椒去腥提味。
    \item 大火煮开后再煮15-20分钟,期间去浮沫。这段时间可以观察肉的上色状况,如果汤汁颜色不够,可以倒一点老抽加深。
    \item 转高压锅压25分钟或转小火焖一小时。期间大约在焖半小时后,就可以先把葱段取出,以免变苦。
    \item 取出香料,尝味调味,不够加盐,如果用的是糖色,可以再提一勺糖,大火收汁。
    \item 盛出撒葱花装饰。
\end{enumerate}

\subsection{创新做法}

\begin{table}[h]
    \centering
    \begin{tabular}{|l||c|}\hline
     \textbf{材料}    &  \textbf{份量}\\ \hline\hline
    五花肉/排骨     &  600 g \\ \hline
    小萝卜     &  适量 \\ \hline
    橘子皮/柚子皮     &  适量 \\ \hline
    糖(最好是冰糖)     &   \\ \hline
    盐     &    \\ \hline
    酱油     &   \\ \hline
    黄酒     &   \\ \hline
    大葱     &  两根 \\ \hline
    生姜     &  两个 \\ \hline
    蒜     &  一颗 \\ \hline
    香菜     &  三四束 \\ \hline
    八角     &  四颗 \\ \hline
    桂皮     &  一段\\ \hline
    白蔻     &  一个\\ \hline
    香叶     &   \\ \hline
    \end{tabular}
    \caption{材料表}
\end{table}

这道菜里的小萝卜也可以换做栗子或其他适合搭荤菜的配菜。

\begin{enumerate}
    \item 肉用葱姜加水泡一晚,去血去腥增香。第二天用清水洗干净放干备用。
    \item 肉切块, 小萝卜洗净去根拍裂(不用变形)。
    \item 炼葱油
    \begin{enumerate}
        \item 锅里倒油,大火烧热。
        \item 准备大蒜半个,剥开拍。 大葱一两根,拍几下切段。洋葱准备四分之一,香菜几束。生姜一个。八角两个,白蔻一个,香叶两三片。
        \item 油温升温过程中先放八角,白蔻,生姜;出香气后放香叶;接着等油温上来,放大葱和洋葱;最后放香菜。按从硬到软的顺序放,以免材料提前焦。
        \item 炸至金黄色后,滤出材料,葱油备用。
    \end{enumerate}
    \item 大火热锅热葱油,下肉煸炒至产生美拉德反应(深褐色,皮微焦),盛出备用。
    \item 中火倒少油入锅,倒适量白糖炒糖色。推荐用水油炒法,冒小泡后倒入沸水炒匀。加一勺料酒提香,盛出备用。
    \item 热锅葱油,后倒入葱段,姜片,蒜煸炒出香味,再放入其余香料(八角,桂皮,香叶)煸炒。
    \item 出香味后倒入肉块继续煸炒。加适量酱油,糖色和黄酒,这个步骤黄酒可以多一点也没事,然后开大火倒入热水直到没过肉。
    \item 煮开以后加盐定味。如果糖色炒的好,不甜,所以再加一勺糖。
    \item 大火煮开后再煮10-15分钟,期间去浮沫。这段时间可以观察肉的上色状况,如果汤汁颜色不够,可以倒一点老抽加深。
    \item 转小火焖一小时。期间大约在焖半小时后,就可以先把葱段取出,以免变苦。
    \begin{itemize}
        \item 期间准备炸小萝卜,推荐在焖40分钟以后再进行此步骤。同时取柚子皮/橘子皮放入红烧肉的锅里,可以去腥提香。
        \item 多撒些细盐到小萝卜身上,入味同时防止小萝卜提前变色。
        \item 热锅热油最高温,下小萝卜炸。炸至熟透,表皮微皱,在变色前盛出备用。
    \end{itemize}
    \item 取出香料,倒入萝卜,大火收汁。
\end{enumerate}


\section{东坡肉}
\begin{table}[h]
    \centering
    \begin{tabular}{|l||c|}\hline
     \textbf{材料}    &  \textbf{份量}\\ \hline\hline
    五花肉     &  1000 g \\ \hline
    糖(最好是冰糖)     &  100g \\ \hline
    酱油     &   \\ \hline
    黄酒     &  一整瓶 \\ \hline
    生姜     &  两三个 \\ \hline
    小葱     &   大量\\ \hline
    \end{tabular}
    \caption{材料表}
\end{table}

\begin{enumerate}
    \item 整块肉冷水下锅加两勺黄酒焯水,去腥去血沫,方便定型。想要皮脆的话,也可改用炸的方式。
    \begin{itemize}
        \item 等焯水的同时准备香料,生姜切片,小葱切段。
    \end{itemize}
    \item 焯好后取出,用刀刮除皮上的杂质,用热水洗干净,切5厘米方块备用。不可用凉水,会导致肉收缩回生,导致烹调时间变长。
    \item 准备一个砂锅,用竹箅子垫底,葱切段垫满锅底,再铺一层姜。五花肉肉皮朝下,挤满。空隙里塞入冰糖,一个18cm直径的砂锅塞满肉加两把冰糖差不多。
    \item 给肉块上均匀淋上生抽。如果冰糖放不多,并且想加深颜色,可以加老抽。看南北口味差异。用好的黄酒倒入砂锅,不要没锅肉,比肉的水平线低一点即可。 不可加水。
    \item 大火煮开后,如有浮沫要滗走。转小火焖60分钟。
    \item 把肉翻面,肉皮朝上,再焖30-60分钟。
    \item 将砂锅里的汤汁过滤备用。将肉放进蒸碗里,淋上少许汤汁,将蒸碗用盖密封。大火蒸30分钟。
    \begin{itemize}
        \item 期间准备一盘清炒小青菜。为了成品外形,小青菜整颗炒,不需改刀。
    \end{itemize}
    \item 将东坡肉置于小青菜上,淋上少许汤汁。
\end{enumerate}

\section{糖醋里脊}

\begin{table}[h]
    \centering
    \begin{tabular}{|l||c|}\hline
     \textbf{材料}    &  \textbf{份量}\\ \hline\hline
    猪里脊    &  \\ \hline
    糖     &   \\ \hline
    盐     &   \\ \hline
    胡椒粉     &   \\ \hline
    料酒     &   \\ \hline
    淀粉     &   \\ \hline
    酱油     &   \\ \hline
    米醋     &   \\ \hline
    生姜     &   \\ \hline
    小葱     &   \\ \hline
    蒜     &  \\ \hline
    \end{tabular}
    \caption{材料表}
\end{table}


\begin{enumerate}
    \item 猪里脊肉切条,加入半勺白胡椒粉,一勺盐和一勺黄酒腌制15到20分钟。
 
    \item 葱姜蒜切末,大约每个约一个大拇指平面的量。
  
    \item 准备糖醋汁。糖醋汁的调法有多种,网上传的比较多是
    1勺料酒,2勺生抽,3勺糖,4勺醋,5勺清水的比例调制糖醋汁。 具体完全可以按个人口味去控制是甜酸还是酸甜口味。 另一种做法是不加水,把糖和醋的比例换一换,直接用 1勺淀粉,微量盐,3勺糖,1勺米醋,少量酱油兑成。米醋多则酸压甜,具体看喜好。然后把之前准备好的葱姜蒜末倒入拌匀。另起一碗水淀粉备用。
    \item 准备挂糊。此处用的是生熟糊。
    \begin{enumerate}
        \item 准备一碗淀粉,倒入大约淀粉一半体积的沸水烫糊。
        \item 准备另一碗淀粉,倒入凉水拌匀。
        \item 将熟糊与生糊混合拌匀。
    \end{enumerate}
    \item 把肉挂糊抓匀。
    \item 中火起油锅,温度到了以后下挂好糊的肉条,不可一次放太多会导致温度下降和颜色不均。
    \item 变色后捞出,捞出稍控凉后再下锅复炸,出锅控油备用。
    \item 中火炒锅加一点底油,倒入糖醋汁炒出糖醋味。如果之前在做糖醋汁时已经兑入淀粉并且汁已经呈糊化状态,则不用加水淀粉。否则可以在这一步倒入一勺水淀粉进行勾芡。
    \item 糖醋汁稠浓后,倒入肉条略微翻炒即可出锅。
\end{enumerate}


\section{日式牛肉盖饭}

\begin{table}[h!]
    \centering
    \begin{tabular}{|l||c|}\hline
     \textbf{材料}    &  \textbf{份量}\\ \hline\hline
    牛五花肉    &  200 g  \\ \hline
    洋葱     &  1个\\ \hline
    姜     &  一拇指块 \\ \hline
    酱油     & 3-4勺  \\ \hline
    味醂     & 2勺  \\ \hline
    清酒     & 3勺  \\ \hline
    白糖     & 1-2勺  \\ \hline
    浓缩日式高汤     & 1勺  \\ \hline
    浓缩日式荞麦面蘸料     & 1勺  \\ \hline
    盐     &  适量 \\ \hline
    米醋     & 1/3勺 \\ \hline
    \end{tabular}
    \caption{材料表}
\end{table}


\begin{enumerate}
    \item 牛肉切片,洋葱切片。
 
    \item 锅烧热下油,将洋葱煸炒出香味后盛出备用。锅内留底油。
  
    \item 下生姜煸香后,倒入适量水。
    
    \item 水烧开后放入牛肉片,用铲子把牛肉推散的同时,把烧出来的杂质浮末用勺子撇干净。
    \item 杂质去除之后,倒入其余所有调味料,放入炒好的洋葱,盖上锅盖焖一到两分钟即可出锅。
\end{enumerate}


\section{日式炸鸡}

根据个人口味,还可以加入椒盐,或其他蘸酱。

\begin{table}[H]
    \centering
    \begin{tabular}{|l||c|}\hline
     \textbf{材料}    &  \textbf{份量}\\ \hline\hline
    鸡腿肉    &  400 g  \\ \hline
    鸡蛋     &  1个\\ \hline
    淀粉     &  适量 \\ \hline
    酱油     & 2勺  \\ \hline
    味醂     & 1/3勺  \\ \hline
    清酒     & 2勺  \\ \hline
    白糖     & 少量  \\ \hline
    生姜     & 1勺  \\ \hline
    蒜     & 1勺  \\ \hline
    盐     &  适量 \\ \hline
    米醋或柠檬汁     & 1/2勺 \\ \hline
    \end{tabular}
    \caption{材料表}
\end{table}


\begin{enumerate}
    \item 鸡肉切块,约5厘米即可。生姜切末,蒜切末。
 
    \item 把酱油,味醂,醋,盐,糖,姜末和蒜末混合成腌酱。
  
    \item 将鸡肉放入腌酱中腌制30到60分钟。
    \begin{itemize}
        \item 快腌好时,将鸡蛋打入一个干净的碗里,淀粉则放在另一个盘里。
    \end{itemize}
    \item 鸡肉腌好后,倒入清酒并搅拌让它吸收均匀,这样可以增加炸好后的水分。
    \item 鸡肉先沾鸡蛋,然后再放入淀粉盘里,提出,拍走多余淀粉。再放入预热好的低温(160~165度/C)炸油里大约2分钟,盛出静置。
    \item 待所有鸡肉静置好后,提高油温至180度,复炸至金黄色即可出锅。
\end{enumerate}


\section{栗子鸡}

\begin{table}[h]
    \centering
    \begin{tabular}{|l||c|}\hline
     \textbf{材料}    &  \textbf{份量}\\ \hline\hline
    鸡肉    &   \\ \hline
    栗子     &  \\ \hline
    葱     &  三根 \\ \hline
    姜     & 一拇指块  \\ \hline
    蒜     & 一两瓣  \\ \hline
    盐     & 适量  \\ \hline
    糖     & 一勺  \\ \hline
    黄酒     & 一两勺  \\ \hline
    酱油     & 两勺  \\ \hline
    老抽     & 一勺  \\ \hline
    \end{tabular}
    \caption{材料表}
\end{table}

\begin{enumerate}
    \item 鸡肉切块,约5厘米即可。葱切段,姜切片。
 
    \item 热锅冷油中火,下葱姜蒜炒香。
  
    \item 下鸡肉翻炒至变色均匀,加入黄酒,酱油,老抽,盐和糖炒至鸡肉均匀上色。

    \item 生栗子可以在这时加入,翻炒一分钟后倒入热水至与鸡肉齐平,不可过多。
    
    \item 大火烧开后用勺子撇去浮末杂质,转小火盖锅盖焖煮20分钟。如果用的是超市的熟栗子,煮一半时再加入。
    
    \item 尝味调味,大火收汁,或作为鸡汤直接出锅。
\end{enumerate}

\section{炒胡萝卜土豆丝}

\begin{table}[h]
    \centering
    \begin{tabular}{|l||c|}\hline
     \textbf{材料}    &  \textbf{份量}\\ \hline\hline
    胡萝卜    &  一根 \\ \hline
    土豆     &  两个 \\ \hline
    葱     &  两根 \\ \hline
    蒜     & 一两瓣  \\ \hline
    盐     & 适量  \\ \hline
    花椒粉     & 少量  \\ \hline
    干辣椒     & 三颗  \\ \hline
    \end{tabular}
    \caption{材料表}
\end{table}

\begin{enumerate}
    \item 胡萝卜切丝,土豆切丝并泡水里去除淀粉,葱切段,蒜切片,辣椒剪段。
 
    \item 土豆滤干水备用。
     
    \item 热锅热油中火,下葱,蒜,辣椒炒香。
  
    \item 下土豆丝和胡萝卜丝爆炒,加盐调味。也可以加入少量花椒粉。
    
    \item 土豆丝变色,一熟便可出锅,这样比较脆。
\end{enumerate}


\section{香蒜孜然猪肉}

\begin{table}[H]
    \centering
    \begin{tabular}{|l||c|}\hline
     \textbf{材料}    &  \textbf{份量}\\ \hline\hline
    带脂肪的猪肉    &   200g\\ \hline
     洋葱    &  1/4 \\ \hline
    牛奶     &  1勺 \\ \hline
    蒜     & 三瓣  \\ \hline
    盐     & 适量  \\ \hline
    花椒粉     & 适量  \\ \hline
    黑椒粉     & 适量  \\ \hline
    辣椒粉     & 适量  \\ \hline
    酱油     & 1勺  \\ \hline
    孜然     & 适量  \\ \hline
    料酒     & 1勺  \\ \hline
    \end{tabular}
    \caption{材料表}
\end{table}

\begin{enumerate}
    \item 猪肉切块或切挑,一指宽。
 
    \item 将猪肉用盐,酱油,料酒,洋葱,牛奶,黑椒粉进行腌制。
     
    \item 热锅热油中火,下蒜炒至变色盛出备用。
  
    \item 转大火,下猪肉翻炒。加入炒好的蒜,孜然,辣椒粉。注意不要炒老了,猪肉表皮起美拉德反应即可。
    
    \item 翻炒并调味后即可出锅。
\end{enumerate}

\section{小鸡炖蘑菇}

\begin{table}[H]
    \centering
    \begin{tabular}{|l||c|}\hline
     \textbf{材料}    &  \textbf{份量}\\ \hline\hline
    鸡    &   \\ \hline
    香菇    &  \\ \hline
    姜     &  \\ \hline
    葱     &  \\ \hline
    糖     & \\ \hline
    盐     &  \\ \hline
    黄酒     &  \\ \hline
    \end{tabular}
    \caption{材料表}
\end{table}

\begin{enumerate}
    \item 鸡肉切块,姜切片,葱切段,香菇放水里泡开。
 
    \item 鸡肉冷水焯水,锅里放葱姜。焯后捞出控水备用。
     
    \item 热锅热油,油少一点,加盐煸炒鸡肉出美拉德反应,盛出备用。
  
    \item 将鸡肉,姜,葱倒入砂锅或煮锅中,并倒入热水和黄酒,大火煮开。
    
    \item 大火煮10分钟,期间用勺子撇去浮沫杂质。倒入香菇转小火焖煮30分钟。根据家里的材料库存,也可以放入枸杞,红枣。
    
    \item 加糖加盐调味,根据对鸡肉的软烂喜好可以继续煮或直接出锅。
\end{enumerate}


\section{微波炉烤茄子}

\begin{table}[H]
    \centering
    \begin{tabular}{|l||c|}\hline
     \textbf{材料}    &  \textbf{份量}\\ \hline\hline
    茄子    &  一个 \\ \hline
    蒜     &  一个 \\ \hline
    葱     &  一两根\\ \hline
    酱油   & 四勺\\ \hline
    蒸鱼豉油 & 两勺 \\ \hline
    辣椒酱 & 一勺 \\ \hline
    花椒粉 & 适量 \\ \hline
    孜然粉 & 适量 \\ \hline
    \end{tabular}
    \caption{材料表}
\end{table}

\begin{enumerate}
    \item 茄子洗干净,蒜切末,葱切葱花。
 
    \item 取一碗调酱汁:倒入四勺酱油(取决于咸度,自调),两勺蒸豆豉油,一勺辣椒酱(可用老干妈),适量花椒粉和孜然粉。
     
    \item 茄子整颗放入微波炉,高火6分钟。根据大小可以缩短。
  
    \item 热锅下油,小火把蒜末炒香。炒香变色后,倒入酱汁,煮开就关火。
    
    \item 加热好的茄子取出,从茄子的尾部往头部切,一分为二,不需要切断。用筷子在茄内部戳小孔用以入味。均匀倒入炒好的酱汁。
    
    \item 将茄子放入微波炉里高火加热6分钟,取出后撒上葱花即可。
\end{enumerate}



\section{清蒸鲈鱼}

\begin{table}[H]
    \centering
    \begin{tabular}{|l||c|}\hline
     \textbf{材料}    &  \textbf{份量}\\ \hline\hline
    鲈鱼    &  1斤 \\ \hline
    大葱    & 两根 \\ \hline
    姜     &  一两块\\ \hline
    红椒   &  一小片\\ \hline
    蒸鱼豉油 &  两勺\\ \hline
    酱油 &  一勺\\ \hline
    香菜  & 一小捆\\ \hline
    盐  & 少量 \\ \hline
    黄酒  &  适量\\ \hline
    \end{tabular}
    \caption{材料表}
\end{table}

\begin{enumerate}
    \item 取一指段大葱切丝,一指段姜切丝,红椒切丝,泡水里使其支棱。
    \item 将鱼处理干净,去鳞去内脏洗净后用厨房纸吸干。在鱼身两侧划三刀,背半开,抹上盐,将姜片和葱段放入鱼肚里,刀口上也可以放上姜片,最后倒上黄酒腌制。
    \item 准备蒸锅,水开后,在蒸盘放入几片生姜和葱段垫底,这样可以使鱼身不挨着蒸盆,使蒸汽流通,均匀加热。鱼拿起来抖一下,因为腌制时会出水,放入蒸盆隔水蒸12分钟。
    \item 取鱼,去掉姜片和葱段放置盘中。将泡好水的姜丝,葱丝和红椒丝均匀摆在鱼身上,香菜摆在周围。
    \item 热锅大火烧热油,180度以上,将热油淋在鱼身上。
    \item 热锅加入两勺蒸鱼豉油和一勺酱油爆沸后,浇在鱼的四周。不能直接淋到鱼,以免破坏鱼的颜色和本身的味道。
\end{enumerate}


\section{韭菜炒鱿鱼}

\begin{table}[H]
    \centering
    \begin{tabular}{|l||c|}\hline
     \textbf{材料}    &  \textbf{份量}\\ \hline\hline
    鱿鱼    &  500克 \\ \hline
    韭菜    & 100克 \\ \hline
    大蒜  &  一瓣\\ \hline
    生姜 & 只有墨鱼需要\\ \hline
    料酒 &  适量 \\ \hline
    酱油 & 适量 \\ \hline
    盐  & 少量 \\ \hline
    \end{tabular}
    \caption{材料表}
\end{table}

\begin{enumerate}
    \item 处理鱿鱼。去掉最外层黑皮,内脏,取软骨,牙齿和吸盘里的锯齿。洗净后鱿鱼身对半切开,鱿鱼腹部内侧45度切花刀,7刀断1次。
    \item 韭菜洗净切4cm段,姜切片,大蒜拍扁。
    \item 烧一锅沸水,快速放入鱿鱼焯水,打卷(3-7秒)即盛出备用。如果用的是墨鱼,它腥味更重,需要在沸水里加入姜片和大蒜。
    \item 热锅热油,下蒜爆香后放入鱿鱼爆炒。加入适量料酒酱油炒1分钟盛出备用。如果是墨鱼,则还需要加入姜片与大蒜一起炒。
    \item 热锅热油,下韭菜爆炒1分钟后,加入鱿鱼再炒30秒。
    \item 调味加盐即可出锅。
\end{enumerate}

\section{椒盐炸虾}

\begin{table}[H]
    \centering
    \begin{tabular}{|l||c|}\hline
     \textbf{材料}    &  \textbf{份量}\\ \hline\hline
    虾    &  500克 \\ \hline
    鸡蛋    & 两个 \\ \hline
    面包糠 & 酥炸需要 \\ \hline
    生姜 & 少量(可选)\\ \hline
    料酒 &  少量 \\ \hline
    椒盐 & 适量 \\ \hline
    玉米淀粉/面粉  &  \\ \hline
    \end{tabular}
    \caption{材料表}
\end{table}

\begin{enumerate}
    \item 虾去壳挑虾线留虾尾洗净,加微量盐洗去外层黏膜。洗净后控干水分备用。
    \item 准备容器,取生姜挤出少许汁,加上料酒,椒盐,加入虾仁进行腌制。
    \item 根据口感可以选择想要的炸法:
    \begin{itemize}
        \item 酥炸:准备一碗鸡蛋液,一碗面包糠,一碗玉米淀粉。虾按顺序均匀沾上玉米淀粉,鸡蛋液,面包糠。
        \item 软炸:准备一碗鸡蛋液,加入微量水澥一下。加入微量盐和面粉拌成面糊。倒入虾仁拌均。
    \end{itemize}
    \item 热锅热油小火3成油温炸虾,大约30秒金黄色后即可捞出。
    \item 所有虾炸完后,将锅里的残渣捞出。中火油温提升后,关火下虾仁复炸10秒即可出锅。
\end{enumerate}



\section{西红柿土豆炖牛腩}

\begin{table}[H]
    \centering
    \begin{tabular}{|l||c|}\hline
     \textbf{材料}    &  \textbf{份量}\\ \hline\hline
    牛腩    &  一斤 \\ \hline
    西红柿    & 两个 \\ \hline
    土豆 & 拳头大小一个 \\ \hline
    胡萝卜 & 一个\\ \hline
    洋葱 & 半个\\ \hline
    大葱 & 一根\\ \hline
    生姜 & 半个\\ \hline
    大蒜 & 3瓣\\ \hline
    八角 & 两个\\ \hline
    桂皮 & 一个\\ \hline
    香叶 & 两片\\ \hline
    料酒 &   适量\\ \hline
    酱油 & 适量 \\ \hline
    冰糖  &  适量\\ \hline
    盐  &  适量\\ \hline
    \end{tabular}
    \caption{材料表}
\end{table}

\begin{enumerate}
    \item 牛腩切块放水里30分钟以上,泡出血水后洗净备用。
    \item 土豆,胡萝卜去皮切滚刀块,西红柿去皮切块,姜切片,洋葱切片,大葱切段。
    \item 准备一口砂锅或煮锅,牛腩冷水下锅加入生姜,料酒后大火煮开,期间撇去浮沫。煮开后用冷水砸三回,把浮末彻底去干净后。取出牛腩备用,去掉生姜,锅里的汤水用来之后直接煮,不倒。
    \item 中小火倒少油入锅炒糖油。
    \item 糖油成型后倒入牛腩煸炒上色。上色后倒入葱段,姜片,蒜瓣,八角,桂皮,香叶继续翻炒。
    \item 出香味后,倒入料酒炒到酒香出来,并且水分快干了,倒入之前的砂锅中。
    \item 砂锅大火煮开后,取出八角桂皮和香叶。加入酱油,慢炖40分钟。
    \item 热锅热油下胡萝卜煸炒,出香味后下洋葱,洋葱味出来后,最后下番茄。缺水时可以用砂锅里的汤汁舀一点过来,以免炒焦。
    \item 将炒好的配菜都加入到砂锅中。继续炖。
    \item 距离水分烧干还剩一小时时,试味道,加盐,酱油调味。再倒入土豆继续炖煮。
    \item 土豆后含有大量淀粉,汤汁会很快变稠,需要注意别烧焦里。烧到土豆已经软了,汤汁稠度刚好即可出锅。
\end{enumerate}


\section{煎豆腐}

\begin{table}[H]
    \centering
    \begin{tabular}{|l||c|}\hline
     \textbf{材料}    &  \textbf{份量}\\ \hline\hline
    豆腐    &  一盒 \\ \hline
    鸡蛋    & 两个 \\ \hline
    蒜 & 3瓣 \\ \hline
    小葱 & 少量 \\ \hline
    盐 &  少量 \\ \hline
    糖 &  少量 \\ \hline
    淀粉 & 适量 \\ \hline
    酱油 & 适量 \\ \hline
    蚝油 & 适量 \\ \hline
    饮用水 & 适量 \\ \hline
    \end{tabular}
    \caption{材料表}
\end{table}

\begin{enumerate}
    \item 豆腐切厚块,小葱切葱花,蒜切末。
    \item 准备酱汁:一勺蒜末,一勺酱油,一勺蚝油,一勺糖,1/4勺盐,一勺淀粉,加入适量饮用水拌匀。
    \item 准备两个碗,鸡蛋打散倒入碗里拌匀,另一个碗里放入淀粉。
    \item 热锅热油小火,取豆腐先裹淀粉,再蘸蛋液,放入锅里慢煎。煎至两面金黄即可盛出备用。
    \item 热锅热油小火,倒入之前煎好的豆腐,再倒入酱汁,慢慢收汁。最后撒上葱花即可出锅
\end{enumerate}

\section{烤鱿鱼}

\begin{table}[H]
    \centering
    \begin{tabular}{|l||c|}\hline
     \textbf{材料}    &  \textbf{份量}\\ \hline\hline
    鱿鱼    &   \\ \hline
    生抽    &  \\ \hline
    蚝油    &  \\ \hline
    盐    &  \\ \hline
    糖    & 少量 \\ \hline
    料酒    &  \\ \hline
    孜然粉    &  \\ \hline
    花椒粉  &  \\ \hline
    辣椒粉    &  \\ \hline
    油    &  \\ \hline
    \end{tabular}
    \caption{材料表}
\end{table}

\begin{enumerate}
    \item 鱿鱼处理完,洗净后吸干水分。
    \item 将所有配料都混成酱汁。
    \item 烤箱220度预热5分钟,不需要完全到达220度。
    \item 鱿鱼刷上酱汁,烤盘铺上锡纸,放入烤箱烤9分钟。
    \item 再刷一遍酱汁,看口味可以再撒辣椒粉和花椒粉。继续烤2分钟出炉。
\end{enumerate}



\section{鸡肉咖喱饭}

\begin{table}[H]
    \centering
    \begin{tabular}{|l||c|}\hline
     \textbf{材料}    &  \textbf{份量}\\ \hline\hline
    鸡腿肉    &   \\ \hline
    咖喱块    &  \\ \hline
    土豆    &  \\ \hline
    洋葱    &  \\ \hline
    胡萝卜    &  \\ \hline
    西兰花    &  \\ \hline
    牛奶    &  \\ \hline
    花生酱    &  \\ \hline
    葱 &  \\ \hline
    姜 &  \\ \hline
    盐 &  \\ \hline
    料酒 &  \\ \hline
    淀粉 &  \\ \hline
    \end{tabular}
    \caption{材料表}
\end{table}

\begin{enumerate}
    \item 鸡腿肉切块,葱切段拍裂,姜切段拍裂。将鸡腿肉和葱姜加上少量料酒,盐和淀粉抓均。腌制20分钟。
    \item 土豆去皮切块并过水(去掉淀粉),胡萝卜去皮切块,洋葱切丝,西兰花扳开备用
    \item 热锅热油小火,下土豆和胡萝卜,炒至开始上色时,下洋葱丝。出香味后呈出。
    \item 转大火下鸡肉翻炒至出梅拉德反应后,倒入炒好的土豆,胡萝卜和洋葱丝。并加入大量水,没过所有材料。
    \item 大火煮开后转小火,将咖哩块放入并保持搅拌,以免糊底。
    \item 咖喱彻底融化,并且土豆和胡萝卜都已经熟透后,转大火开始收汁。这个步骤时倒入少许牛奶和一勺花生酱提味。
    \item 收汁快结束时,加入西兰花。
    \item 汤汁浓稠后即可出锅。
\end{enumerate}


\section{烤花菜}

\begin{table}[H]
    \centering
    \begin{tabular}{|l||c|}\hline
     \textbf{材料}    &  \textbf{份量}\\ \hline\hline
    花菜    &   \\ \hline
    盐    &  2 \\ \hline
    花椒粉    & 0.5 \\ \hline
    辣椒粉    & 1.5 \\ \hline
    孜然    &  \\ \hline
    油    &  4 \\ \hline
    \end{tabular}
    \caption{材料表}
\end{table}

\begin{enumerate}
    \item 将除了花菜以外的材料混合均匀
    \item 花菜蘸上婚后和的材料,放置烤盘中
    \item 烤箱预热5分钟,200度烤15分钟,表面起梅拉德反应即可关闭
\end{enumerate}



\section{烤鸡腿}

\begin{table}[H]
    \centering
    \begin{tabular}{|l||c|}\hline
     \textbf{材料}    &  \textbf{份量}\\ \hline\hline
    鸡腿    &   \\ \hline
    盐    &  少量 \\ \hline
    料酒    & 少量 \\ \hline
    酱油    & 适量 \\ \hline
    花椒粉    & 适量 \\ \hline
    胡椒粉    & 适量 \\ \hline
    孜然    & 少量 \\ \hline
    糖    & 少量 \\ \hline
    姜    & 数片 \\ \hline
    大蒜    & 少量 \\ \hline
    油    &  少量 \\ \hline
    \end{tabular}
    \caption{材料表}
\end{table}

\begin{enumerate}
    \item 鸡腿需要切几刀,清洗并放掉血水后,用厨房纸擦干
    \item 材料混合,鸡腿腌制2小时
    \begin{itemize}
        \item 若想做类似奥尔良烤鸡翅的味道的话,将花椒,胡椒,孜然和糖替换为蜂蜜即可
    \end{itemize}
    \item 鸡腿抹上油放置烤盘上,若鸡腿油脂丰富,可以不用放油
    \item 烤箱预热5分钟,200度烤20分钟,翻面后再烤20分钟
\end{enumerate}


\section{麻婆豆腐}

\begin{table}[H]
    \centering
    \begin{tabular}{|l||c|}\hline
     \textbf{材料}    &  \textbf{份量}\\ \hline\hline
    卤水豆腐/嫩豆腐    &  350g  \\ \hline
    青蒜苗    &  少量 \\ \hline
    牛肉    & 50g \\ \hline
    郫县豆瓣    & 20g \\ \hline
    豆豉    & 5g \\ \hline
    辣椒粉    & 3g \\ \hline
    花椒粉    & 1g \\ \hline
    盐    & 少量 \\ \hline
    高汤/水 & 150g \\ \hline
    料酒    & 5g \\ \hline
    酱油    & 10g \\ \hline
    水淀粉    &  25g \\ \hline
    \end{tabular}
    \caption{材料表}
\end{table}

\begin{enumerate}
    \item 准备材料:牛肉切末,豆豉剁碎,蒜苗切菱形,豆腐切成麻将大小的方块
    \item 豆腐飞水:一锅热水微沸后倒入些许盐,轻动作放入豆腐,中小火两分钟后滤水备用。
    \item 热锅热油中火,煸炒牛肉末,炒酥至梅拉德反应后捞出备用
    \item 留底油转小火,加入豆瓣炒出红油后,加入豆豉和辣椒粉,炒出香味后加入高汤或水,酱油,盐和料酒,转中火
    \item 烧开以后放入豆腐,轻晃豆腐使其均匀受热
    \item 当汤汁到豆腐的一半水位时,加入青蒜苗和牛肉末推匀,并用水淀粉(1:1.5)进行第一次勾芡,待水转沸以后再进行第二次勾芡。
    \item 试味,整体勾芡完成后,撒上花椒粉出锅
\end{enumerate}


\section{拔丝系列}

\begin{table}[H]
    \centering
    \begin{tabular}{|l||c|}\hline
     \textbf{材料}    &  \textbf{份量}\\ \hline\hline
    苹果/香蕉/红薯    &  350g  \\ \hline
    淀粉    &   \\ \hline
    清水    &  \\ \hline
    绵白糖    &  \\ \hline
    白醋 & \\ \hline 
    \end{tabular}
    \caption{材料表}
\end{table}

\begin{enumerate}
    \item 将主食材去皮去核后切块。若食材本身难熟,比如红薯,可以先煮再炸。不需煮透
    \item 食材加淀粉抹均匀
    \item 热锅多热油,六成热时将主食材入锅炸至金黄色,捞出控油
    \item 另起一锅,热锅水油炒糖,冒小泡浅黄色即可
    \item 倒入炸好的主食材迅速翻均,倒入微量醋后,用手蘸水,轻微洒水降温,一边翻炒一边洒水。直到均匀裹上糖浆后即可出锅
    \item 出锅后可以趁热迅速将料理拔丝裹成一圈。
    
\end{enumerate}
需注意此菜非常烫,但又需要趁热吃,应配一碗凉水上桌。在吃之前将拔丝蘸一下凉水再入口


\section{地三鲜}

\begin{table}[H]
    \centering
    \begin{tabular}{|l||c|}\hline
     \textbf{材料}    &  \textbf{份量}\\ \hline\hline
    茄子    &   \\ \hline
    土豆    &   \\ \hline
    青椒    &  \\ \hline
    生姜 & 三片\\ \hline 
    大葱 & 少量\\ \hline 
    大蒜 & 大量\\ \hline 
    盐 & \\ \hline 
    酱油 & \\ \hline 
    水淀粉 & \\ \hline 
    开水 & \\ \hline 
    \end{tabular}
    \caption{材料表}
\end{table}

\begin{enumerate}
    \item 茄子洗净带皮切滚刀块,土豆去皮切滚刀块,青椒去籽切片。
    \item 葱姜切粒,蒜单独切粒。蒜分两半,一半与葱姜混合,一半留至上菜前使用。
    \item 热锅冷油下土豆煎至内部透明,表面金黄后出锅备用
    \item 热锅倒大量油,下茄子煎至金黄出油后出锅备用
    \item 用之前的底油,下青椒快速翻炒至出香味出锅备用
    \item 之前炒好的茄子,土豆和青椒可用滤网进行控油
    \item 热锅少冷油小火,下葱姜蒜炝香后倒入酱油出香味后倒入开水
    \item 水沸后倒入之前炒好的土豆,青椒和茄子。加入两勺盐,一勺糖提鲜。 小火炒制。
    \item 调味完成后,在出锅前淋水淀粉勾芡。倒入水淀粉后需轻晃锅,使其均匀上芡。出锅前开大火亮汁
    \item 出锅装盘后,撒上生蒜末完成
\end{enumerate}


\section{辣子鸡}

\begin{table}[H]
    \centering
    \begin{tabular}{|l||c|}\hline
     \textbf{材料}    &  \textbf{份量}\\ \hline\hline
    鸡腿肉   &   \\ \hline
    盐    & 两勺  \\ \hline
    酱油    & 三勺 \\ \hline
    糖 & 一勺\\ \hline 
    干辣椒 & 大量\\ \hline 
    胡椒粉 & 适量 \\ \hline
    花椒粒 & 适量\\ \hline 
    八角 &  一颗\\ \hline 
    葱 & 适量\\ \hline 
    蒜 & 适量\\ \hline 
    \end{tabular}
    \caption{材料表}
\end{table}

\begin{enumerate}
    \item 干辣椒剪段,葱切段,蒜切片
    \item 鸡腿肉去骨,切花刀后切成球块
    \item 对鸡腿进行腌制:先加入胡椒粉,料酒,一小勺盐和少量酱油(提色)抓均。然后加入淀粉抓均后封油。腌制15分钟
    \item 下锅油炸前在鸡腿中倒入一勺香油
    \item 热锅大量油中火,7成热时下鸡腿肉炸至金黄色后出锅
    \item 复炸,共炸三次,第三次油炸后,鸡块成枣红色
    \item 另起一锅,热锅少冷油小火,倒入半勺香油,倒入花椒,八角煸香,再下大量干辣椒
    \item 出香气后加入鸡块转大火翻炒。此时进行调味,一勺糖,一勺料酒,两勺酱油。
    \item 调味完成后加入葱段蒜片翻炒均匀后出锅
\end{enumerate}




\section{菠萝炒饭}

\begin{table}[H]
    \centering
    \begin{tabular}{|l||c|}\hline
     \textbf{材料}    &  \textbf{份量}\\ \hline\hline
    剩米饭   &   \\ \hline
    菠萝   &   \\ \hline
    胡萝卜    &  \\ \hline
    青豆 & \\ \hline 
    腊肠 & \\ \hline 
    鸡蛋 &  \\ \hline
    盐 & \\ \hline 
    胡椒粉 &  \\ \hline
    料酒 & \\ \hline 
    \end{tabular}
    \caption{材料表}
\end{table}

\begin{enumerate}
    \item 菠萝肉取出切丁,胡萝卜和腊肠切丁
    \item 鸡蛋加少许盐,微量胡椒和料酒打匀。
    \item 热锅无油小火,下腊肠炒至出油后倒入胡萝卜翻炒至快熟后,倒入青豆炒熟后出锅备用
    \item 热锅无油小火,将菠萝快速翻炒至出水后,滤水出锅备用。这一目的是为了防止菠萝和米饭翻炒时,大量出水导致米饭过于黏稠。
    \item 热锅热油,下鸡蛋快速翻炒至成型即可出锅备用
    \item 热锅热油中火,下米饭翻炒。倒入少量料酒翻炒使米饭均匀分开,松。
    \item 米饭炒松后,加盐调味。倒入前面的材料,转大火快速翻炒。炒至味道融合后出锅
    \item 可以将炒好后的米饭放入菠萝壳中,慢蒸5分钟
\end{enumerate}


\section{蛋炒饭}

\begin{table}[H]
    \centering
    \begin{tabular}{|l||c|}\hline
     \textbf{材料}    &  \textbf{份量}\\ \hline\hline
    剩米饭   &   \\ \hline
    胡萝卜    &  \\ \hline
    青豆 & \\ \hline 
    腊肠 & \\ \hline 
    鸡蛋 &  \\ \hline
    小葱 & \\ \hline
    盐 & \\ \hline 
    酱油 &  \\ \hline 
    料酒 & \\ \hline 
    \end{tabular}
    \caption{材料表}
\end{table}

\begin{enumerate}
    \item 胡萝卜和腊肠切丁,小葱切葱花,
    \item 鸡蛋加少许盐和料酒打匀。
    \item 热锅热油,下鸡蛋快速翻炒至成型即可出锅备用
    \item 热锅无油小火,下腊肠炒至出油后倒入胡萝卜翻炒至快熟后,倒入青豆炒熟后出锅备用
    \item 热锅热油中火,下米饭翻炒。倒入适量料酒翻炒使米饭均匀分开,松。倒入少量酱油提鲜。
    \item 米饭炒松后,加盐调味。倒入前面的材料,转大火快速翻炒。炒至味道融合,米饭不沾锅
    \item 出锅前撒上葱花盖上锅盖焖一下,出香气后出锅
\end{enumerate}


\section{黄金炒饭}

\begin{table}[H]
    \centering
    \begin{tabular}{|l||c|}\hline
     \textbf{材料}    &  \textbf{份量}\\ \hline\hline
    剩米饭   &  1人份 \\ \hline
    胡萝卜    &  \\ \hline
    青豆 & \\ \hline 
    鸡蛋 &  3个\\ \hline
    盐 & \\ \hline 
    \end{tabular}
    \caption{材料表}
\end{table}

\begin{enumerate}
    \item 胡萝切丁,鸡蛋蛋黄蛋白分离。
    \item 蛋黄与米饭混合均匀
    \item 热锅热水,将青豆,胡萝卜丁焯水后滤水备用
    \item 热锅热油小火,将蛋白煎至成型后出锅备用
    \item 热锅热油(葱油最佳)小火,下拌好后的米饭炒至表面蛋黄凝固后,加盐调味
    \item 调味完成后,倒入蛋白,青豆和胡萝卜丁,转中火翻炒出香气后出锅
\end{enumerate}


\section{葱油饼}

\begin{table}[H]
    \centering
    \begin{tabular}{|l||c|}\hline
     \textbf{材料}    &  \textbf{份量}\\ \hline\hline
    中筋面粉   &  300g \\ \hline
    沸水    & 150g  \\ \hline
    油 & 1勺 \\ \hline 
    葱 &  大量\\ \hline
    胡椒 & 半勺\\ \hline 
    盐 & 半勺\\ \hline 
    \end{tabular}
    \caption{材料表}
\end{table}

\begin{enumerate}
    \item 葱切末
    \item 面粉过筛,将面粉和水,揉成面团,放置醒面。
    \item 醒后可将面团分成两个小面团。将面团擀至大圆片,抹上色拉油后,将胡椒,葱末和盐均匀撒在面皮上。然后卷成长条状,再绕成盘旋状,最后将收口压紧,静置3-5分钟,将盘旋状面团擀成圆饼就可以了
    \item 另一种做法为将面团擀成正方形。跟上一步一样撒上材料。在面皮上切四刀(九宫格四周),然后从旁边开始叠起,最终叠成方块,再整成圆形擀成圆饼。
    \item 热锅热少油小火,放入葱油饼煎至两面金黄。需要盖锅盖,翻边内部熟透。
\end{enumerate}


\section{排骨糯米饭(待修正)}

\begin{table}[H]
    \centering
    \begin{tabular}{|l||c|}\hline
     \textbf{材料}    &  \textbf{份量}\\ \hline\hline
    糯米   &   \\ \hline
    排骨    &   \\ \hline
    冬笋 &  \\ \hline 
    青豆 & \\ \hline
    酱油 &  \\ \hline
    盐 & \\ \hline 
    \end{tabular}
    \caption{材料表}
\end{table}

\begin{enumerate}
    \item 糯米提前洗净待用
    \item 佐料都提前烹饪,比如排骨可以用红烧排骨的方式先做。同时确保味道调整完毕
    \item 准备一口锅,倒糯米和佐料,加水至水平线。
    \item 大火煮到沸腾,开小火。全程关盖
    \item 水干即可
\end{enumerate}


\section{盐焗鸡}

\begin{table}[H]
    \centering
    \begin{tabular}{|l||c|}\hline
     \textbf{材料}    &  \textbf{份量}\\ \hline\hline
    鸡   &   \\ \hline
    生姜    &   \\ \hline
    盐 &  \\ \hline 
    水 & \\ \hline
    料酒 &  \\ \hline
    葱 & \\ \hline 
    蒜 & \\ \hline 
    \end{tabular}
    \caption{材料表}
\end{table}

\begin{enumerate}
    \item 整只鸡洗干净,内外抹盐。内部置入生姜
    \item 放入电饭煲焖煮。不需要加水或盐
    \item 时间到后打开,确认鸡都熟的差不多了,倒入半碗水+酒。两次
    \item 完成后,用鸡汤加姜蒜末+酱油+醋调成蘸汁
\end{enumerate}



\section{新疆大盘鸡}

\begin{table}[H]
    \centering
    \begin{tabular}{|l||c|}\hline
     \textbf{材料}    &  \textbf{份量}\\ \hline\hline
    鸡   &   \\ \hline
    土豆    &   \\ \hline
    洋葱 &  \\ \hline 
    青椒 & \\ \hline
    干红辣椒 &  \\ \hline
    花椒 & \\ \hline 
    八角 & \\ \hline 
    豆瓣酱 & \\ \hline 
    姜 & \\ \hline 
    蒜& \\ \hline 
    酱油 & \\ \hline 
    盐 & \\ \hline 
    料酒 & \\ \hline 
    啤酒 & \\ \hline 
    糖 & \\ \hline 
    \end{tabular}
    \caption{材料表}
\end{table}

\begin{enumerate}
    \item 鸡肉切块,土豆削皮滚刀块,青椒切块,洋葱切块
    \item 锅里烧水,水开之后放2片姜,料酒,2粒花椒。下鸡肉焯水,1分钟后捞出
    \item 热锅热油小火,下花椒、八角、干红辣椒,炸香
    \item 下姜蒜,炒出香味后放入鸡肉,加料酒,转大火炒至鸡肉变色
    \item 加一勺豆瓣酱,翻炒均匀,把土豆放进去,加一些酱油,一起翻炒至上色
    \item 倒一厅啤酒,要没过鸡肉,大火烧开后转小火煮,煮到汤只剩下1/3多一点的时候放入洋葱,翻炒均匀继续煮
    \item 炖到汁收的差不多,放入青椒,蒜末,盐翻炒,加一点点水焖(这里的水就是最后的汤的量)
    \item 收汁后出锅
\end{enumerate}






\section{担担面}

\begin{table}[H]
    \centering
    \begin{tabular}{|l||c|}\hline
     \textbf{材料}    &  \textbf{份量}\\ \hline\hline
    猪肉   &   \\ \hline
    新鲜面条 & \\ \hline
    宜宾芽菜    &   \\ \hline
    姜 &  \\ \hline 
    花椒 & \\ \hline
    料酒 &  \\ \hline
    老抽 & \\ \hline 
    酱油 & \\ \hline
    糖 & \\ \hline 
    盐 & \\ \hline
    辣椒油 & \\ \hline
    花椒油 & \\ \hline
    醋 & 少量\\ \hline
    葱 & \\ \hline
    蒜 & \\ \hline
    花生 & \\ \hline
    \end{tabular}
    \caption{材料表}
\end{table}

\begin{enumerate}
    \item 猪肉洗净,擦干水份,剁成肉糜,加盐、姜碎拌匀腌15钟左右
    \item 热锅热油小火,下姜片爆香后捞出姜片。下猪肉炒散
    \item 猪肉炒至变色后,沿锅边倒入料酒
    \item 转中火导入老抽上色后,下宜宾芽菜翻炒至出香味
    \item 下花椒,将猪肉炒至梅拉德反应,下糖炒均出锅备用
    \item 将酱油,辣椒油,花椒油,醋,糖,葱碎,蒜碎混合至面碗中
    \item 煮锅烧开水,下面条煮熟,出锅后过冷水,放入装有调料的碗中
    \item 撒上炒好的芽菜肉末和花生碎
\end{enumerate}



\section{虾仁滑蛋}

\begin{table}[H]
    \centering
    \begin{tabular}{|l||c|}\hline
     \textbf{材料}    &  \textbf{份量}\\ \hline\hline
    虾仁   &   \\ \hline
    鸡蛋 & 4个\\ \hline
    小葱    &   \\ \hline
    料酒 & \\ \hline
    蛋白 & \\ \hline
    淀粉 & \\ \hline 
    白胡椒粉 & \\ \hline
    \end{tabular}
    \caption{材料表}
\end{table}

\begin{enumerate}
    \item 虾去壳去线,削去表面红皮去腥后。虾仁用少许盐混合均匀放置5分钟,然后将盐洗去(用盐先抓洗一下就可以洗去表面的黏液,这样炒出来的虾仁才会清脆好吃);用厨房纸将水份吸干虾仁,如果没有吸干水份,腌制的时候不容易附着腌料
    \item 擦干的虾仁加盐、料酒、胡椒粉,蛋白搅拌均匀后,再下玉米淀粉拌匀腌制片刻
    \item 鸡蛋打散加入调味料(盐1/2小勺、白胡椒粉少许)混合均匀,香葱切成葱花,水淀粉搅拌均匀准备好;烧一锅滚水,将腌拌好的虾仁入锅,略为汆烫至七分熟(约30秒),备用
    \item 将烫过的虾仁放入蛋液中,加入葱花、水淀粉混合均匀
    \item 起油锅,放入2大勺油,小火,油烧热后,将混合虾仁的蛋汁倒入,一开始不要翻动,感觉蛋液周围开始凝固就用锅铲迅速划圈圈将蛋炒散,等大部份的蛋液一凝固马上关火盛盘,避免蛋炒的太熟不够滑嫩
\end{enumerate}


\section{爆炒蛭子}

\begin{table}[H]
    \centering
    \begin{tabular}{|l||c|}\hline
     \textbf{材料}    &  \textbf{份量}\\ \hline\hline
    蛭子   &   \\ \hline
    葱 & \\ \hline
    姜    &   \\ \hline
    蒜 & \\ \hline
    豆豉酱 & \\ \hline
    青椒 & \\ \hline 
    红椒 & \\ \hline
    洋葱 & \\ \hline
    \end{tabular}
    \caption{材料表}
\end{table}

\begin{enumerate}
    \item 蛏子放盐水中浸泡至少两小时吐沙子
    \item 葱姜蒜切片,洋葱切块,青椒红椒切块备用
    \item 泡好的蛏子放入沸水中焯水捞出备用
    \item 油锅烧热,葱姜蒜炒出香味,依次放入洋葱青椒红椒,最后放适量豆豉酱
    \item 蛏子放入锅中翻炒(豆豉酱本身就有咸味,无需再放盐)
\end{enumerate}


\section{白斩鸡}

\begin{table}[H]
    \centering
    \begin{tabular}{|l||c|}\hline
     \textbf{材料}    &  \textbf{份量}\\ \hline\hline
    鸡   &   \\ \hline
    葱 & \\ \hline
    姜    &   \\ \hline
    八角 & 少量\\ \hline
    冰块 & \\ \hline
    盐 & \\ \hline 
    鱼露 & \\ \hline
    酱油 & \\ \hline
    米酒/料酒 & \\ \hline
    醋 & \\ \hline
    \end{tabular}
    \caption{材料表}
\end{table}

\begin{enumerate}
    \item 葱切段,姜切片,鸡肉洗干净
    \item 热锅热油小火,下葱段姜片炒出香味,表面起梅拉德反应后加入八角关火焖一分钟后出锅备用
    \item 起一口煮锅下到足够掩盖鸡肉的水,大火烧开。
    \item 水开后,放入炒好的调料。三起三放后转小火焖煮。 鸡腿的话是20分钟,整鸡视大小,最多不能超过30分钟。
    \item 准备一个盆,倒入冰块。如果没有就用冰水代替。 倒入鱼露(如果没有,用酱油代替),米酒,盐巴。
    \item 鸡拿出后直接放入冰盆里。因为热胀冷缩原理,鸡肉会快速吸收在冰块里的调料。省去了抹盐的步骤。
    \item 最后煮锅里的鸡汤可以盛出做蘸酱。 蘸酱按照个人喜好调制,简单的就是鸡汤加酱油和醋调和即可
\end{enumerate}

推荐取出鸡肉后,去骨。以免骨血染在鸡肉上。然后把骨头继续放在煮锅里煮。这样就顺便完成了一道鸡肉高汤


\section{炸小酥肉}

\begin{table}[H]
    \centering
    \begin{tabular}{|l||c|}\hline
     \textbf{材料}    &  \textbf{份量}\\ \hline\hline
    坐板肉   &   \\ \hline
    红薯粉 & \\ \hline
    花椒粒  &   \\ \hline
    菜籽油 & \\ \hline
    葱 & \\ \hline
    姜 & \\ \hline 
    黄酒  & \\ \hline
    胡椒粉 & \\ \hline
    香油 & \\ \hline
    盐 & \\ \hline
    \end{tabular}
    \caption{材料表}
\end{table}

\begin{enumerate}
    \item 肉改刀脱皮切条备用
    \item 葱切段拍打,姜切片拍打,通过拍打出汁,放入碗中加水和黄酒调成葱姜水备用
    \item 腌制猪肉:猪肉加盐,葱姜水,胡椒粉, 抓均。混合完成后加入少量香油增香。
    \item 热锅无油小火焙花椒,焙香后磨成花椒面
    \item 将花椒面和红薯粉与猪肉混合挂糊。揉均匀后倒入菜籽油方便起酥。
    \item 热锅热油中火,放筷子后能起泡则温度达标。下酥肉炸至金黄色后捞出滤油。
    \item 复炸两次:每炸一次,适当提高一点油温,以此找到刚好酥脆的状态。需注意每次炸完应将酥肉滤油,以免回软
    \item 炸完后可配剩余焙好的花椒面和盐做椒盐蘸吃
\end{enumerate}


\section{手擀面}

\begin{table}[H]
    \centering
    \begin{tabular}{|l||c|}\hline
     \textbf{材料}    &  \textbf{份量}\\ \hline\hline
    面粉 & 170g \\ \hline
    鸡蛋 + 水 &  80g \\ \hline
    盐 & 1g \\ \hline
    \end{tabular}
    \caption{材料表}
\end{table}

\begin{enumerate}
    \item 材料为一人份
    \item 温水揉面至三光后,醒面2小时
\end{enumerate}


\section{面筋}

\begin{table}[H]
    \centering
    \begin{tabular}{|l||c|}\hline
     \textbf{材料}    &  \textbf{份量}\\ \hline\hline
    面粉 & 200g \\ \hline
    水 &  110g \\ \hline
    盐 & 1g  \\ \hline
    \end{tabular}
    \caption{材料表}
\end{table}

\begin{enumerate}
    \item  揉成面团醒30分钟
    \item 再揉到三光后醒1小时,再揉再醒1小时
    \item 放入水中多次冲洗,洗出面筋
    \item 洗出后放入水中醒30分钟,揉成圆团
\end{enumerate}


\section{萝卜炖牛肉}

\begin{table}[H]
    \centering
    \begin{tabular}{|l||c|}\hline
     \textbf{材料}    &  \textbf{份量}\\ \hline\hline
    牛肉 &  \\ \hline
    萝卜 &   \\ \hline
    花椒 & \\ \hline
    料酒 & \\ \hline
    香叶 & \\ \hline
    八角 & \\ \hline
    桂皮 & \\ \hline
    生姜 & \\ \hline
    泡椒片 & \\ \hline
    老抽 & \\ \hline
    啤酒 & \\ \hline
    淀粉 & \\ \hline
    葱 & \\ \hline
    香菜 & \\ \hline
    蚝油 & \\ \hline
    盐 & \\ \hline
    胡椒粉 & \\ \hline
    \end{tabular}
    \caption{材料表}
\end{table}

\begin{enumerate}
    \item 牛肉切块侵泡水中40分钟。去血水
    \item 白萝卜切块,姜切片,小葱切段,八角香叶桂皮过水清洗
    \item 牛肉冷水下锅加料酒花椒焯水后捞出备用
    \item 热锅下油,5成热后下花椒,香叶,八角。桂皮炒香,加入牛肉。3分钟炒香后加入生姜和泡椒片(可选)。翻炒后加料酒,蚝油,老抽。
    \item 材料炒上色后加入热水,啤酒一瓶。大火到水沸腾后,去浮沫,开小火焖煮20-30分钟后,取出八角,香叶,桂皮。继续焖煮30分钟。加盐,胡椒调味搅拌后加入水淀粉勾薄芡,加小葱段,香菜后出锅
\end{enumerate}