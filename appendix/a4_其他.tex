\section{小技巧}

\subsection{油炸时的温度控制}

\begin{table}[h]
    \centering
    \begin{tabular}{|l||c|}\hline
     \textbf{温度}    &  \textbf{作用}\\ \hline\hline
    90-130     &  滑炒 \\ \hline
    130-170    &   炸熟。 比如炸鸡的第一次炸制时,先用这个温度炸熟\\ \hline
    170-230     &   炸菜,上色,定型。 比如松鼠鱼,脆皮,复炸 \\ \hline
    
    \end{tabular}
    \caption{油炸温度控制表}
\end{table}


\subsection{荤菜去腥臭}
可以在最后烹调阶段加入柚子皮或橘皮进行去腥。

\subsection{淀粉和面粉挂糊的区别}
淀粉脆,面粉韧。

\subsection{淀粉之间的用途区别}
基本上玉米淀粉是最全能的。

玉米淀粉用来勾芡油炸,颜色好味道香

土豆淀粉用来嫩肉做酱,颜色透明光亮

红薯淀粉用作粉皮油炸,口感滑韧干爽

\subsection{甜点装饰}
部分甜品撒一点跳跳粉,吃起来有惊喜。



\subsection{谷氨酸钠}
鸡精主要成分。番茄含有大量


\subsection{戚风烤制注意事项}
1、放中层还是下层?——放下层。
2、为什么会分层,底部出现布丁层?——蛋白霜和芝士面糊翻拌不均匀会出现布丁层。
3、为什么会开裂?——温度过高或者蛋白打发太硬就容易开裂,可以试下及时降温。
4、烤不熟拿出来后,可以回炉再烤吗?——如果刚拿出来就发现不熟,可以立刻回炉再烤,但会回缩,如果出炉已经晾凉一段时间了,就不要回炉了,会变硬的。
5、长不高或者回缩厉害是什么原因?——蛋白没打发到位会长不高,蛋白打发过硬长高太快就会回缩厉害。
6、收腰是什么原因?——是还没烤熟。